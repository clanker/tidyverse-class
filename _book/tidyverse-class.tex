\documentclass[]{book}
\usepackage{lmodern}
\usepackage{amssymb,amsmath}
\usepackage{ifxetex,ifluatex}
\usepackage{fixltx2e} % provides \textsubscript
\ifnum 0\ifxetex 1\fi\ifluatex 1\fi=0 % if pdftex
  \usepackage[T1]{fontenc}
  \usepackage[utf8]{inputenc}
\else % if luatex or xelatex
  \ifxetex
    \usepackage{mathspec}
  \else
    \usepackage{fontspec}
  \fi
  \defaultfontfeatures{Ligatures=TeX,Scale=MatchLowercase}
\fi
% use upquote if available, for straight quotes in verbatim environments
\IfFileExists{upquote.sty}{\usepackage{upquote}}{}
% use microtype if available
\IfFileExists{microtype.sty}{%
\usepackage{microtype}
\UseMicrotypeSet[protrusion]{basicmath} % disable protrusion for tt fonts
}{}
\usepackage{hyperref}
\hypersetup{unicode=true,
            pdftitle={Hitchhiker's Guide to the Tidyverse (and Statistical Learning in R)},
            pdfauthor={Cory Lanker},
            pdfborder={0 0 0},
            breaklinks=true}
\urlstyle{same}  % don't use monospace font for urls
\usepackage{natbib}
\bibliographystyle{apalike}
\usepackage{color}
\usepackage{fancyvrb}
\newcommand{\VerbBar}{|}
\newcommand{\VERB}{\Verb[commandchars=\\\{\}]}
\DefineVerbatimEnvironment{Highlighting}{Verbatim}{commandchars=\\\{\}}
% Add ',fontsize=\small' for more characters per line
\usepackage{framed}
\definecolor{shadecolor}{RGB}{248,248,248}
\newenvironment{Shaded}{\begin{snugshade}}{\end{snugshade}}
\newcommand{\KeywordTok}[1]{\textcolor[rgb]{0.13,0.29,0.53}{\textbf{#1}}}
\newcommand{\DataTypeTok}[1]{\textcolor[rgb]{0.13,0.29,0.53}{#1}}
\newcommand{\DecValTok}[1]{\textcolor[rgb]{0.00,0.00,0.81}{#1}}
\newcommand{\BaseNTok}[1]{\textcolor[rgb]{0.00,0.00,0.81}{#1}}
\newcommand{\FloatTok}[1]{\textcolor[rgb]{0.00,0.00,0.81}{#1}}
\newcommand{\ConstantTok}[1]{\textcolor[rgb]{0.00,0.00,0.00}{#1}}
\newcommand{\CharTok}[1]{\textcolor[rgb]{0.31,0.60,0.02}{#1}}
\newcommand{\SpecialCharTok}[1]{\textcolor[rgb]{0.00,0.00,0.00}{#1}}
\newcommand{\StringTok}[1]{\textcolor[rgb]{0.31,0.60,0.02}{#1}}
\newcommand{\VerbatimStringTok}[1]{\textcolor[rgb]{0.31,0.60,0.02}{#1}}
\newcommand{\SpecialStringTok}[1]{\textcolor[rgb]{0.31,0.60,0.02}{#1}}
\newcommand{\ImportTok}[1]{#1}
\newcommand{\CommentTok}[1]{\textcolor[rgb]{0.56,0.35,0.01}{\textit{#1}}}
\newcommand{\DocumentationTok}[1]{\textcolor[rgb]{0.56,0.35,0.01}{\textbf{\textit{#1}}}}
\newcommand{\AnnotationTok}[1]{\textcolor[rgb]{0.56,0.35,0.01}{\textbf{\textit{#1}}}}
\newcommand{\CommentVarTok}[1]{\textcolor[rgb]{0.56,0.35,0.01}{\textbf{\textit{#1}}}}
\newcommand{\OtherTok}[1]{\textcolor[rgb]{0.56,0.35,0.01}{#1}}
\newcommand{\FunctionTok}[1]{\textcolor[rgb]{0.00,0.00,0.00}{#1}}
\newcommand{\VariableTok}[1]{\textcolor[rgb]{0.00,0.00,0.00}{#1}}
\newcommand{\ControlFlowTok}[1]{\textcolor[rgb]{0.13,0.29,0.53}{\textbf{#1}}}
\newcommand{\OperatorTok}[1]{\textcolor[rgb]{0.81,0.36,0.00}{\textbf{#1}}}
\newcommand{\BuiltInTok}[1]{#1}
\newcommand{\ExtensionTok}[1]{#1}
\newcommand{\PreprocessorTok}[1]{\textcolor[rgb]{0.56,0.35,0.01}{\textit{#1}}}
\newcommand{\AttributeTok}[1]{\textcolor[rgb]{0.77,0.63,0.00}{#1}}
\newcommand{\RegionMarkerTok}[1]{#1}
\newcommand{\InformationTok}[1]{\textcolor[rgb]{0.56,0.35,0.01}{\textbf{\textit{#1}}}}
\newcommand{\WarningTok}[1]{\textcolor[rgb]{0.56,0.35,0.01}{\textbf{\textit{#1}}}}
\newcommand{\AlertTok}[1]{\textcolor[rgb]{0.94,0.16,0.16}{#1}}
\newcommand{\ErrorTok}[1]{\textcolor[rgb]{0.64,0.00,0.00}{\textbf{#1}}}
\newcommand{\NormalTok}[1]{#1}
\usepackage{longtable,booktabs}
\usepackage{graphicx,grffile}
\makeatletter
\def\maxwidth{\ifdim\Gin@nat@width>\linewidth\linewidth\else\Gin@nat@width\fi}
\def\maxheight{\ifdim\Gin@nat@height>\textheight\textheight\else\Gin@nat@height\fi}
\makeatother
% Scale images if necessary, so that they will not overflow the page
% margins by default, and it is still possible to overwrite the defaults
% using explicit options in \includegraphics[width, height, ...]{}
\setkeys{Gin}{width=\maxwidth,height=\maxheight,keepaspectratio}
\IfFileExists{parskip.sty}{%
\usepackage{parskip}
}{% else
\setlength{\parindent}{0pt}
\setlength{\parskip}{6pt plus 2pt minus 1pt}
}
\setlength{\emergencystretch}{3em}  % prevent overfull lines
\providecommand{\tightlist}{%
  \setlength{\itemsep}{0pt}\setlength{\parskip}{0pt}}
\setcounter{secnumdepth}{5}
% Redefines (sub)paragraphs to behave more like sections
\ifx\paragraph\undefined\else
\let\oldparagraph\paragraph
\renewcommand{\paragraph}[1]{\oldparagraph{#1}\mbox{}}
\fi
\ifx\subparagraph\undefined\else
\let\oldsubparagraph\subparagraph
\renewcommand{\subparagraph}[1]{\oldsubparagraph{#1}\mbox{}}
\fi

%%% Use protect on footnotes to avoid problems with footnotes in titles
\let\rmarkdownfootnote\footnote%
\def\footnote{\protect\rmarkdownfootnote}

%%% Change title format to be more compact
\usepackage{titling}

% Create subtitle command for use in maketitle
\providecommand{\subtitle}[1]{
  \posttitle{
    \begin{center}\large#1\end{center}
    }
}

\setlength{\droptitle}{-2em}

  \title{Hitchhiker's Guide to the Tidyverse (and Statistical Learning in R)}
    \pretitle{\vspace{\droptitle}\centering\huge}
  \posttitle{\par}
    \author{Cory Lanker}
    \preauthor{\centering\large\emph}
  \postauthor{\par}
      \predate{\centering\large\emph}
  \postdate{\par}
    \date{2019-07-28}

\usepackage{booktabs}
\usepackage{amsthm}
\makeatletter
\def\thm@space@setup{%
  \thm@preskip=8pt plus 2pt minus 4pt
  \thm@postskip=\thm@preskip
}
\makeatother

\begin{document}
\maketitle

{
\setcounter{tocdepth}{1}
\tableofcontents
}
\chapter*{Introduction}\label{introduction}
\addcontentsline{toc}{chapter}{Introduction}

This bookdown notebook can be cloned via

\begin{Shaded}
\begin{Highlighting}[]
\FunctionTok{git}\NormalTok{ clone git@github.com:clanker/tidyverse-class.git}
\end{Highlighting}
\end{Shaded}

Introducing the tidyverse analyzing these data sets:

\begin{enumerate}
\def\labelenumi{\arabic{enumi}.}
\tightlist
\item
  Basic plots with \texttt{tibble} and \texttt{ggplot2} using
  \texttt{Boston} house prices.
\item
  Preprocessing with \texttt{tidyr} and \texttt{dplyr} using
  \texttt{Lahman} baseball data.
\end{enumerate}

\textbf{Other useful packages}

Though some of these commands will be used, we won't go deeply into the
following tidyverse packages. These packages have an obvious function
space, so knowing when to use these packages and how to find the
appropriate function is easier than the packages discussed here.

\begin{enumerate}
\def\labelenumi{\arabic{enumi}.}
\tightlist
\item
  Reading in data with \texttt{readr}.
\item
  String manipulation with \texttt{stringr}.
\item
  Dates and times with \texttt{lubridate}.
\item
  Handling factors with \texttt{forcats}.
\item
  Apply functions with \texttt{purrr}.
\end{enumerate}

Some good ways to learn about these packages:

\begin{itemize}
\tightlist
\item
  \texttt{vignette()}, and search for documentation of that package,
\item
  the cheat sheets for the packages on the RStudio website, or
\item
  example(``function'') for helpful guidance on usage.
\end{itemize}

R proficiency is assumed. These notes aim to bring a functional R coder
into the tidyverse realm for modern data analysis.

\begin{Shaded}
\begin{Highlighting}[]
\CommentTok{# To install the necessary packages in the tidyverse:}
\KeywordTok{install.packages}\NormalTok{(}\StringTok{"tidyverse"}\NormalTok{, }\DataTypeTok{dependencies =} \OtherTok{TRUE}\NormalTok{)}
\end{Highlighting}
\end{Shaded}

\textbf{to do list}

\begin{enumerate}
\def\labelenumi{\arabic{enumi}.}
\tightlist
\item
  Add Chapter: computing using \texttt{caret}.
\item
  Add Chapter: functions provided by \texttt{mlr}.
\item
  Add Chapter: implementing \texttt{keras}.
\item
  Add Chapter: tips and tricks for better \texttt{R} coding.
\end{enumerate}

Many references are made to Hadley Wickham's book, \emph{R for Data
Science} \citep{wickham2016r}. This document is built with R Markdown,
\textbf{knitr} \citep{xie2015}, and the \textbf{bookdown} package
\citep{R-bookdown}.

\chapter{\texorpdfstring{tibbles, ggplot2, and the
\emph{tidyverse}}{tibbles, ggplot2, and the tidyverse}}\label{ch:intro}

The tidyverse universe includes:

In general, the tidyverse is the following:

\begin{enumerate}
\def\labelenumi{\arabic{enumi}.}
\tightlist
\item
  provided the \texttt{pipe} command \texttt{\%\textgreater{}\%}
\end{enumerate}

\begin{itemize}
\tightlist
\item
  \texttt{x\ \%\textgreater{}\%\ f(y,\ z,\ ...)} is
  \texttt{f(x,\ y,\ z,\ ...)}
\item
  allows chained commands for better coherence
\item
  e.g., \texttt{mtcars\ \%\textgreater{}\%\ apply(2,\ mean)} is error
  without \texttt{tidyr::\%\textgreater{}\%}
\end{itemize}

\begin{enumerate}
\def\labelenumi{\arabic{enumi}.}
\setcounter{enumi}{1}
\tightlist
\item
  \texttt{tibble} is the improved data structure of the tidyverse
\end{enumerate}

\begin{itemize}
\tightlist
\item
  easier to read-in data to a useful format
\item
  automatic type conversion
\item
  nicer printing options
\end{itemize}

\begin{enumerate}
\def\labelenumi{\arabic{enumi}.}
\setcounter{enumi}{2}
\tightlist
\item
  \texttt{dplyr} provides tibble manipulation commands
\end{enumerate}

\begin{itemize}
\tightlist
\item
  understandable data processing with \texttt{pipe} streams
\item
  \textbf{filter} data faster
\item
  \textbf{arrange} rows of data easily
\item
  \textbf{select} columns quickly
\item
  \textbf{mutate} variables
\item
  \textbf{summarize} according to \texttt{group\_by()}
\item
  also provides SQL relational operations
\end{itemize}

\begin{enumerate}
\def\labelenumi{\arabic{enumi}.}
\setcounter{enumi}{3}
\tightlist
\item
  \texttt{ggplot2} is a plotting syntax (grammar of graphics)
\end{enumerate}

\begin{itemize}
\tightlist
\item
  \texttt{qplot()} provides a sensible \textbf{q}uick \textbf{plot}
\item
  apply plot types to data rather than the reverse
\item
  e.g.
  \texttt{ggplot(data)\ +\ plot\_type(aes(xvar,\ yvar,\ groups),\ options)}
\item
  allows grid of plots by group using \textbf{facets}
\item
  overlays statistical summaries, e.g. \texttt{+\ geom\_smooth(x,\ y)}
\item
  ``add'' options such as transformed axes, labels, coordinates, etc.
\end{itemize}

\begin{enumerate}
\def\labelenumi{\arabic{enumi}.}
\setcounter{enumi}{4}
\tightlist
\item
  \texttt{readr} is a faster, less painful read-in method
\end{enumerate}

\begin{itemize}
\tightlist
\item
  \texttt{read\_fun} denotes \texttt{readr} functions (instead of
  \texttt{read.fun})
\item
  guesses column types
\item
  offers writing functions, too
\item
  allows read and write with RDS, R's binary format
\end{itemize}

\begin{enumerate}
\def\labelenumi{\arabic{enumi}.}
\setcounter{enumi}{5}
\tightlist
\item
  \texttt{tidyr} recharacterizes tibbles
\end{enumerate}

\begin{itemize}
\tightlist
\item
  \texttt{spread()} turns key and value columns into key-category
  columns
\item
  e.g., \texttt{state,\ year,\ pop} into
  \texttt{state,\ 1990,\ 1991,\ ...} of pop values
\item
  \texttt{gather()} turns expands data frames by condensing columns
\item
  e.g., condenses \texttt{1990,\ 1991,\ ...} into two
  \texttt{year,\ pop} columns
\end{itemize}

\begin{enumerate}
\def\labelenumi{\arabic{enumi}.}
\setcounter{enumi}{6}
\tightlist
\item
  Other helpful tidyverse packages:
\end{enumerate}

\begin{itemize}
\tightlist
\item
  \texttt{stringr} offers many useful \texttt{str\_fun} operations
\item
  \texttt{forcats} has operations \_for cat\_egorical variables
\item
  \texttt{lubridate} provides date and time control
\item
  \texttt{purrr}
\end{itemize}

The examples I'll use in the next few chapters are the Boston housing
database and the Lahman baseball database. By doing analysis on these
two data sets, I hope to introduce the power of the tidyverse.

\section{Tibbles: Boston housing
data}\label{tibbles-boston-housing-data}

Load, convert, print a tibble.

\begin{Shaded}
\begin{Highlighting}[]
\CommentTok{# Convert to a tibble so it prints nicely}
\KeywordTok{library}\NormalTok{(MASS)}
\NormalTok{select <-}\StringTok{ }\NormalTok{dplyr}\OperatorTok{::}\NormalTok{select}
\NormalTok{boston <-}\StringTok{ }\KeywordTok{as_tibble}\NormalTok{(MASS}\OperatorTok{::}\NormalTok{Boston)}
\NormalTok{boston}
\end{Highlighting}
\end{Shaded}

\begin{verbatim}
## # A tibble: 506 x 14
##       crim    zn indus  chas   nox    rm   age   dis   rad   tax ptratio
##      <dbl> <dbl> <dbl> <int> <dbl> <dbl> <dbl> <dbl> <int> <dbl>   <dbl>
##  1 0.00632  18    2.31     0 0.538  6.58  65.2  4.09     1   296    15.3
##  2 0.0273    0    7.07     0 0.469  6.42  78.9  4.97     2   242    17.8
##  3 0.0273    0    7.07     0 0.469  7.18  61.1  4.97     2   242    17.8
##  4 0.0324    0    2.18     0 0.458  7.00  45.8  6.06     3   222    18.7
##  5 0.0690    0    2.18     0 0.458  7.15  54.2  6.06     3   222    18.7
##  6 0.0298    0    2.18     0 0.458  6.43  58.7  6.06     3   222    18.7
##  7 0.0883   12.5  7.87     0 0.524  6.01  66.6  5.56     5   311    15.2
##  8 0.145    12.5  7.87     0 0.524  6.17  96.1  5.95     5   311    15.2
##  9 0.211    12.5  7.87     0 0.524  5.63 100    6.08     5   311    15.2
## 10 0.170    12.5  7.87     0 0.524  6.00  85.9  6.59     5   311    15.2
## # ... with 496 more rows, and 3 more variables: black <dbl>, lstat <dbl>,
## #   medv <dbl>
\end{verbatim}

\begin{Shaded}
\begin{Highlighting}[]
\NormalTok{?MASS}\OperatorTok{::}\NormalTok{Boston}
\end{Highlighting}
\end{Shaded}

\begin{itemize}
\tightlist
\item
  crim per capita crime rate by town.
\item
  zn proportion of residential land zoned for lots over 25,000 sq.ft.
\item
  indus proportion of non-retail business acres per town.
\item
  chas Charles River dummy variable (= 1 if tract bounds river; 0
  otherwise).
\item
  nox nitrogen oxides concentration (parts per 10 million).
\item
  rm average number of rooms per dwelling.
\item
  age proportion of owner-occupied units built prior to 1940.
\item
  dis weighted mean of distances to five Boston employment centres.
\item
  rad index of accessibility to radial highways.
\item
  tax full-value property-tax rate per \$10,000.
\item
  ptratio pupil-teacher ratio by town.
\item
  black \(1000(Bk - 0.63)^2\) where Bk is the proportion of blacks by
  town.
\item
  lstat lower status of the population (percent).
\item
  medv median value of owner-occupied homes in \$1000s.
\end{itemize}

A ggplot is the first declaration (usually variable \texttt{data} is
defined), followed by graphics definitions (operations on the data):

\begin{Shaded}
\begin{Highlighting}[]
\KeywordTok{ggplot}\NormalTok{(}\DataTypeTok{data =}\NormalTok{ boston) }\OperatorTok{+}\StringTok{ }
\StringTok{  }\KeywordTok{geom_point}\NormalTok{(}\DataTypeTok{mapping =} \KeywordTok{aes}\NormalTok{(}\DataTypeTok{x =}\NormalTok{ rm, }\DataTypeTok{y =}\NormalTok{ medv), }\DataTypeTok{alpha=}\FloatTok{0.4}\NormalTok{) }\OperatorTok{+}
\StringTok{  }\KeywordTok{labs}\NormalTok{(}\DataTypeTok{x =} \StringTok{"avg. rooms per house"}\NormalTok{,}
       \DataTypeTok{y =} \StringTok{"median house value"}\NormalTok{,}
       \DataTypeTok{title =} \StringTok{"House values vs. size in Boston"}\NormalTok{)}
\end{Highlighting}
\end{Shaded}

\includegraphics{tidyverse-class_files/figure-latex/price-rooms-1-1.pdf}

Making a histogram of all numeric variables. First step, gather all
variables.

\begin{Shaded}
\begin{Highlighting}[]
\NormalTok{boston }\OperatorTok
\StringTok{  }\KeywordTok{keep}\NormalTok{(is.numeric) }\OperatorTok\StringTok{  }\CommentTok{# strips all non-numeric columns (unnecessary here)}
\StringTok{  }\KeywordTok{gather}\NormalTok{()  }\CommentTok{# puts all variable values in a single column 'value'}
\end{Highlighting}
\end{Shaded}

\begin{verbatim}
## # A tibble: 7,084 x 2
##    key     value
##    <chr>   <dbl>
##  1 crim  0.00632
##  2 crim  0.0273 
##  3 crim  0.0273 
##  4 crim  0.0324 
##  5 crim  0.0690 
##  6 crim  0.0298 
##  7 crim  0.0883 
##  8 crim  0.145  
##  9 crim  0.211  
## 10 crim  0.170  
## # ... with 7,074 more rows
\end{verbatim}

Facet wrap allows plotting each \texttt{key} level separately.

\begin{Shaded}
\begin{Highlighting}[]
\NormalTok{boston }\OperatorTok\StringTok{ }
\StringTok{  }\KeywordTok{gather}\NormalTok{() }\OperatorTok
\StringTok{  }\KeywordTok{ggplot}\NormalTok{() }\OperatorTok{+}
\StringTok{    }\KeywordTok{facet_wrap}\NormalTok{(}\OperatorTok{~}\StringTok{ }\NormalTok{key, }\DataTypeTok{scales =} \StringTok{"free"}\NormalTok{) }\OperatorTok{+}
\StringTok{    }\KeywordTok{geom_histogram}\NormalTok{(}\DataTypeTok{mapping =} \KeywordTok{aes}\NormalTok{(value), }\DataTypeTok{bins=}\DecValTok{20}\NormalTok{)}
\end{Highlighting}
\end{Shaded}

\includegraphics{tidyverse-class_files/figure-latex/all-var-hist-1.pdf}

From the histograms, there seems to be only a few values of
\texttt{crim} over 30.

\begin{Shaded}
\begin{Highlighting}[]
\NormalTok{boston }\OperatorTok
\StringTok{  }\KeywordTok{filter}\NormalTok{(crim }\OperatorTok{>}\StringTok{ }\DecValTok{30}\NormalTok{)}
\end{Highlighting}
\end{Shaded}

\begin{verbatim}
## # A tibble: 8 x 14
##    crim    zn indus  chas   nox    rm   age   dis   rad   tax ptratio black
##   <dbl> <dbl> <dbl> <int> <dbl> <dbl> <dbl> <dbl> <int> <dbl>   <dbl> <dbl>
## 1  89.0     0  18.1     0 0.671  6.97  91.9  1.42    24   666    20.2 397. 
## 2  38.4     0  18.1     0 0.693  5.45 100    1.49    24   666    20.2 397. 
## 3  41.5     0  18.1     0 0.693  5.53  85.4  1.61    24   666    20.2 329. 
## 4  67.9     0  18.1     0 0.693  5.68 100    1.43    24   666    20.2 385. 
## 5  51.1     0  18.1     0 0.597  5.76 100    1.41    24   666    20.2   2.6
## 6  45.7     0  18.1     0 0.693  4.52 100    1.66    24   666    20.2  88.3
## 7  73.5     0  18.1     0 0.679  5.96 100    1.80    24   666    20.2  16.4
## 8  37.7     0  18.1     0 0.679  6.20  78.7  1.86    24   666    20.2  18.8
## # ... with 2 more variables: lstat <dbl>, medv <dbl>
\end{verbatim}

\section{ggplot2 and EDA}\label{ggplot2-and-eda}

But we want to know the conditional distributions according to
\texttt{medv}. First, showing this with conditional densities.

\begin{Shaded}
\begin{Highlighting}[]
\NormalTok{boston }\OperatorTok\StringTok{ }
\StringTok{  }\KeywordTok{gather}\NormalTok{(}\StringTok{'key'}\NormalTok{, }\StringTok{'value'}\NormalTok{, }\OperatorTok{-}\NormalTok{medv) }\OperatorTok
\StringTok{  }\KeywordTok{mutate}\NormalTok{(}\DataTypeTok{price_gr =} \KeywordTok{ntile}\NormalTok{(medv, }\DecValTok{4}\NormalTok{)) }\OperatorTok
\StringTok{  }\KeywordTok{ggplot}\NormalTok{(}\KeywordTok{aes}\NormalTok{(value, }\DataTypeTok{group =}\NormalTok{ price_gr)) }\OperatorTok{+}
\StringTok{    }\KeywordTok{facet_wrap}\NormalTok{(}\OperatorTok{~}\StringTok{ }\NormalTok{key, }\DataTypeTok{ncol =} \DecValTok{4}\NormalTok{, }\DataTypeTok{scales =} \StringTok{"free"}\NormalTok{) }\OperatorTok{+}
\StringTok{    }\KeywordTok{geom_freqpoly}\NormalTok{(}\KeywordTok{aes}\NormalTok{(}\DataTypeTok{color =}\NormalTok{ price_gr), }\DataTypeTok{bins =} \DecValTok{12}\NormalTok{)}
\end{Highlighting}
\end{Shaded}

\includegraphics{tidyverse-class_files/figure-latex/cond-hist-1.pdf}

\begin{Shaded}
\begin{Highlighting}[]
\CommentTok{# Click on the expand icon at the top right to make bigger.}
\end{Highlighting}
\end{Shaded}

Appears \texttt{chas} is categorical.

\begin{Shaded}
\begin{Highlighting}[]
\NormalTok{boston <-}\StringTok{ }\NormalTok{boston }\OperatorTok
\StringTok{  }\KeywordTok{mutate}\NormalTok{(}\DataTypeTok{chas =} \KeywordTok{factor}\NormalTok{(chas))}
\end{Highlighting}
\end{Shaded}

Second, scatterplots of median value vs.~all variables.

\begin{Shaded}
\begin{Highlighting}[]
\NormalTok{boston }\OperatorTok
\StringTok{  }\KeywordTok{gather}\NormalTok{(}\StringTok{'key'}\NormalTok{, }\StringTok{'value'}\NormalTok{, }\OperatorTok{-}\KeywordTok{c}\NormalTok{(medv, chas)) }\OperatorTok\StringTok{ }
\StringTok{  }\KeywordTok{ggplot}\NormalTok{(}\KeywordTok{aes}\NormalTok{(}\DataTypeTok{x =}\NormalTok{ value, }\DataTypeTok{y =}\NormalTok{ medv)) }\OperatorTok{+}
\StringTok{    }\KeywordTok{facet_wrap}\NormalTok{( }\OperatorTok{~}\StringTok{ }\NormalTok{key, }\DataTypeTok{scales =} \StringTok{"free"}\NormalTok{) }\OperatorTok{+}
\StringTok{    }\KeywordTok{geom_point}\NormalTok{(}\KeywordTok{aes}\NormalTok{(}\DataTypeTok{shape =}\NormalTok{ chas), }\DataTypeTok{size =} \FloatTok{0.5}\NormalTok{, }\DataTypeTok{alpha =} \FloatTok{0.25}\NormalTok{) }\OperatorTok{+}
\StringTok{    }\KeywordTok{geom_smooth}\NormalTok{(}\DataTypeTok{lwd =} \FloatTok{0.5}\NormalTok{, }\DataTypeTok{se =} \OtherTok{TRUE}\NormalTok{) }\OperatorTok{+}
\StringTok{  }\KeywordTok{ggsave}\NormalTok{(}\StringTok{'plots/medv-scatter.pdf'}\NormalTok{)}
\end{Highlighting}
\end{Shaded}

\begin{verbatim}
## Saving 6.5 x 4.5 in image
\end{verbatim}

\begin{verbatim}
## `geom_smooth()` using method = 'loess' and formula 'y ~ x'
## `geom_smooth()` using method = 'loess' and formula 'y ~ x'
\end{verbatim}

\includegraphics{tidyverse-class_files/figure-latex/medv-scatter-1.pdf}

\begin{Shaded}
\begin{Highlighting}[]
\CommentTok{# Click on the expand icon at the top right to make bigger.}
\end{Highlighting}
\end{Shaded}

There are ggplot jitter options, but none worked for me.

\begin{Shaded}
\begin{Highlighting}[]
\NormalTok{boston }\OperatorTok
\StringTok{  }\KeywordTok{gather}\NormalTok{(}\StringTok{'key'}\NormalTok{, }\StringTok{'value'}\NormalTok{, }\OperatorTok{-}\KeywordTok{c}\NormalTok{(medv, chas)) }\OperatorTok\StringTok{ }
\StringTok{  }\KeywordTok{ggplot}\NormalTok{(}\KeywordTok{aes}\NormalTok{(}\DataTypeTok{x =}\NormalTok{ value, }\DataTypeTok{y =}\NormalTok{ medv)) }\OperatorTok{+}
\StringTok{    }\KeywordTok{facet_wrap}\NormalTok{( }\OperatorTok{~}\StringTok{ }\NormalTok{key, }\DataTypeTok{scales =} \StringTok{"free"}\NormalTok{) }\OperatorTok{+}
\StringTok{    }\KeywordTok{geom_jitter}\NormalTok{(}\KeywordTok{aes}\NormalTok{(}\DataTypeTok{shape =}\NormalTok{ chas), }\DataTypeTok{size =} \FloatTok{0.5}\NormalTok{, }\DataTypeTok{alpha =} \FloatTok{0.25}\NormalTok{) }\OperatorTok{+}\StringTok{ }
\StringTok{    }\KeywordTok{geom_smooth}\NormalTok{(}\DataTypeTok{lwd =} \FloatTok{0.5}\NormalTok{, }\DataTypeTok{se =} \OtherTok{TRUE}\NormalTok{)}
\end{Highlighting}
\end{Shaded}

\begin{verbatim}
## `geom_smooth()` using method = 'loess' and formula 'y ~ x'
\end{verbatim}

\includegraphics{tidyverse-class_files/figure-latex/medv-jitter-nw-1.pdf}

Tinkering to get a jittered plot.

\begin{Shaded}
\begin{Highlighting}[]
\NormalTok{var_sd <-}\StringTok{ }\NormalTok{boston }\OperatorTok
\StringTok{  }\KeywordTok{gather}\NormalTok{(}\StringTok{'key'}\NormalTok{, }\StringTok{'value'}\NormalTok{, }\OperatorTok{-}\KeywordTok{c}\NormalTok{(medv, chas)) }\OperatorTok\StringTok{ }
\StringTok{  }\KeywordTok{group_by}\NormalTok{(key) }\OperatorTok\StringTok{ }
\StringTok{  }\KeywordTok{summarize}\NormalTok{(}\DataTypeTok{var_sd =} \KeywordTok{sd}\NormalTok{(value))}
\NormalTok{boston }\OperatorTok
\StringTok{  }\KeywordTok{gather}\NormalTok{(}\StringTok{'key'}\NormalTok{, }\StringTok{'value'}\NormalTok{, }\OperatorTok{-}\KeywordTok{one_of}\NormalTok{(}\KeywordTok{c}\NormalTok{(}\StringTok{"medv"}\NormalTok{, }\StringTok{"chas"}\NormalTok{))) }\OperatorTok\StringTok{ }
\StringTok{  }\KeywordTok{left_join}\NormalTok{(}\DataTypeTok{y =}\NormalTok{ var_sd, }\DataTypeTok{by =} \StringTok{"key"}\NormalTok{) }\OperatorTok\StringTok{ }
\StringTok{  }\KeywordTok{mutate}\NormalTok{(}\DataTypeTok{jit_val =}\NormalTok{ value }\OperatorTok{+}\StringTok{ }\NormalTok{var_sd }\OperatorTok{*}\StringTok{ }\KeywordTok{runif}\NormalTok{(}\KeywordTok{nrow}\NormalTok{(boston), }\OperatorTok{-}\FloatTok{0.1}\NormalTok{, }\FloatTok{0.1}\NormalTok{)) }\OperatorTok\StringTok{ }
\StringTok{  }\KeywordTok{ggplot}\NormalTok{(}\KeywordTok{aes}\NormalTok{(}\DataTypeTok{x =}\NormalTok{ jit_val, }\DataTypeTok{y =}\NormalTok{ medv)) }\OperatorTok{+}
\StringTok{    }\KeywordTok{facet_wrap}\NormalTok{( }\OperatorTok{~}\StringTok{ }\NormalTok{key, }\DataTypeTok{scales =} \StringTok{"free"}\NormalTok{) }\OperatorTok{+}
\StringTok{    }\KeywordTok{geom_jitter}\NormalTok{(}\KeywordTok{aes}\NormalTok{(}\DataTypeTok{shape =}\NormalTok{ chas), }\DataTypeTok{size =} \FloatTok{0.5}\NormalTok{, }\DataTypeTok{alpha =} \FloatTok{0.25}\NormalTok{) }\OperatorTok{+}
\StringTok{    }\KeywordTok{geom_smooth}\NormalTok{(}\DataTypeTok{lwd =} \FloatTok{0.5}\NormalTok{, }\DataTypeTok{se =} \OtherTok{TRUE}\NormalTok{) }\OperatorTok{+}
\StringTok{  }\KeywordTok{ggsave}\NormalTok{(}\StringTok{'plots/medv-jitter.pdf'}\NormalTok{)}
\end{Highlighting}
\end{Shaded}

\begin{verbatim}
## Saving 6.5 x 4.5 in image
\end{verbatim}

\begin{verbatim}
## `geom_smooth()` using method = 'loess' and formula 'y ~ x'
## `geom_smooth()` using method = 'loess' and formula 'y ~ x'
\end{verbatim}

\includegraphics{tidyverse-class_files/figure-latex/medv-jitter-1.pdf}

Covariance plot of variables.

\begin{Shaded}
\begin{Highlighting}[]
\NormalTok{boston }\OperatorTok\StringTok{ }
\StringTok{  }\KeywordTok{keep}\NormalTok{(is.numeric) }\OperatorTok\StringTok{ }
\StringTok{  }\KeywordTok{cor}\NormalTok{() }\OperatorTok\StringTok{ }
\StringTok{  }\KeywordTok{as_tibble}\NormalTok{() }\OperatorTok\StringTok{ }
\StringTok{  }\KeywordTok{mutate}\NormalTok{(}\DataTypeTok{name =} \KeywordTok{colnames}\NormalTok{(boston[}\KeywordTok{sapply}\NormalTok{(boston, is.numeric)])) }\OperatorTok\StringTok{ }
\StringTok{  }\KeywordTok{gather}\NormalTok{( , , }\OperatorTok{-}\KeywordTok{one_of}\NormalTok{(}\StringTok{"name"}\NormalTok{)) }\OperatorTok\StringTok{ }
\StringTok{  }\KeywordTok{ggplot}\NormalTok{(}\KeywordTok{aes}\NormalTok{(name, key, }\DataTypeTok{fill =}\NormalTok{ value)) }\OperatorTok{+}
\StringTok{    }\KeywordTok{geom_tile}\NormalTok{() }\OperatorTok{+}
\StringTok{    }\KeywordTok{scale_fill_gradient2}\NormalTok{(}\DataTypeTok{low =} \StringTok{"blue"}\NormalTok{, }\DataTypeTok{mid =} \StringTok{"white"}\NormalTok{, }\DataTypeTok{high =} \StringTok{"red"}\NormalTok{,}
                         \DataTypeTok{breaks =} \KeywordTok{seq}\NormalTok{(}\OperatorTok{-}\DecValTok{1}\NormalTok{, }\DecValTok{1}\NormalTok{, , }\DataTypeTok{by =} \FloatTok{0.2}\NormalTok{)) }\OperatorTok{+}
\StringTok{    }\KeywordTok{theme}\NormalTok{(}\DataTypeTok{legend.key.height =} \KeywordTok{unit}\NormalTok{(}\DecValTok{45}\NormalTok{, }\StringTok{"pt"}\NormalTok{))}
\end{Highlighting}
\end{Shaded}

\includegraphics{tidyverse-class_files/figure-latex/cov-plot-1.pdf} But
a better correlation plot is in a package designed for them.

\begin{Shaded}
\begin{Highlighting}[]
\KeywordTok{library}\NormalTok{(corrplot)}
\NormalTok{boston }\OperatorTok\StringTok{ }
\StringTok{  }\KeywordTok{keep}\NormalTok{(is.numeric) }\OperatorTok\StringTok{ }
\StringTok{  }\KeywordTok{cor}\NormalTok{() }\OperatorTok
\StringTok{  }\KeywordTok{abs}\NormalTok{() }\OperatorTok\StringTok{ }
\StringTok{  }\KeywordTok{corrplot}\NormalTok{(}\DataTypeTok{cl.lim =} \KeywordTok{c}\NormalTok{(}\DecValTok{0}\NormalTok{, }\DecValTok{1}\NormalTok{))}
\end{Highlighting}
\end{Shaded}

\includegraphics{tidyverse-class_files/figure-latex/using-corrplot-1.pdf}

Analyze median value and highway access \texttt{rad}.

\begin{Shaded}
\begin{Highlighting}[]
\NormalTok{boston }\OperatorTok
\StringTok{  }\KeywordTok{ggplot}\NormalTok{(}\KeywordTok{aes}\NormalTok{(rad, medv)) }\OperatorTok{+}
\StringTok{  }\KeywordTok{geom_jitter}\NormalTok{(}\KeywordTok{aes}\NormalTok{(}\DataTypeTok{color =}\NormalTok{ chas), }
              \DataTypeTok{height =} \DecValTok{2}\NormalTok{, }\DataTypeTok{width =} \OtherTok{NULL}\NormalTok{, }
              \DataTypeTok{size =} \DecValTok{1}\NormalTok{, }\DataTypeTok{alpha =} \FloatTok{0.4}\NormalTok{) }\OperatorTok{+}
\StringTok{  }\KeywordTok{geom_smooth}\NormalTok{(}\KeywordTok{aes}\NormalTok{(}\DataTypeTok{color =}\NormalTok{ chas), }\DataTypeTok{lwd =} \DecValTok{1}\NormalTok{)}
\end{Highlighting}
\end{Shaded}

\begin{verbatim}
## `geom_smooth()` using method = 'loess' and formula 'y ~ x'
\end{verbatim}

\includegraphics{tidyverse-class_files/figure-latex/medv-rad-1.pdf}

Perhaps \texttt{rad\ =\ 24} is a missing value.

\begin{Shaded}
\begin{Highlighting}[]
\NormalTok{boston }\OperatorTok
\StringTok{  }\KeywordTok{count}\NormalTok{(rad)}
\end{Highlighting}
\end{Shaded}

\begin{verbatim}
## # A tibble: 9 x 2
##     rad     n
##   <int> <int>
## 1     1    20
## 2     2    24
## 3     3    38
## 4     4   110
## 5     5   115
## 6     6    26
## 7     7    17
## 8     8    24
## 9    24   132
\end{verbatim}

\begin{Shaded}
\begin{Highlighting}[]
\NormalTok{boston }\OperatorTok
\StringTok{  }\KeywordTok{gather}\NormalTok{( , , }\OperatorTok{-}\NormalTok{rad) }\OperatorTok\StringTok{ }
\StringTok{  }\KeywordTok{group_by}\NormalTok{(key, rad) }\OperatorTok
\StringTok{  }\KeywordTok{mutate}\NormalTok{(}\DataTypeTok{value =} \KeywordTok{as.numeric}\NormalTok{(value)) }\OperatorTok\StringTok{  }\CommentTok{# necessary due to factor variable chas}
\StringTok{  }\KeywordTok{summarize}\NormalTok{(}\DataTypeTok{z =} \KeywordTok{round}\NormalTok{(}\KeywordTok{mean}\NormalTok{(value), }\DecValTok{1}\NormalTok{)) }\OperatorTok\StringTok{ }
\StringTok{  }\KeywordTok{spread}\NormalTok{(rad, z)}
\end{Highlighting}
\end{Shaded}

\begin{verbatim}
## Warning: attributes are not identical across measure variables;
## they will be dropped
\end{verbatim}

\begin{verbatim}
## # A tibble: 13 x 10
## # Groups:   key [13]
##    key       `1`   `2`   `3`   `4`   `5`   `6`   `7`   `8`  `24`
##    <chr>   <dbl> <dbl> <dbl> <dbl> <dbl> <dbl> <dbl> <dbl> <dbl>
##  1 age      45    64.8  49.3  60.8  69.2  60.1  40.1  67.3  89.8
##  2 black   389.  386.  392.  383.  369.  387.  388.  385.  288. 
##  3 chas      0     0     0.1   0.1   0.1   0     0     0.2   0.1
##  4 crim      0     0.1   0.1   0.4   0.7   0.2   0.2   0.4  12.8
##  5 dis       6     4.1   5.1   4.4   3.7   4     6.5   4.4   2.1
##  6 indus     5.1   9.6   4.4  10.7   9.8   8.2   5     5.9  18.1
##  7 lstat     7.4  10     9.1  12.2  10.7  12.3   8     8    18.6
##  8 medv     24.4  26.8  27.9  21.4  25.7  21    27.1  30.4  16.4
##  9 nox       0.5   0.5   0.5   0.5   0.6   0.5   0.4   0.5   0.7
## 10 ptratio  17.6  17.3  18.2  19.1  16.5  17.8  18.4  18    20.2
## 11 rm        6.6   6.6   6.5   6.1   6.4   6.1   6.6   7     6  
## 12 tax     291.  261.  246.  336   332.  373.  304.  301.  666  
## 13 zn       39.9  20.4  16.4  14.7  11.1  13    26.7   6.2   0
\end{verbatim}

Or in helpful boxplot format.

\begin{Shaded}
\begin{Highlighting}[]
\NormalTok{boston }\OperatorTok
\StringTok{  }\KeywordTok{keep}\NormalTok{(is.numeric) }\OperatorTok\StringTok{ }
\StringTok{  }\KeywordTok{gather}\NormalTok{( , , }\OperatorTok{-}\NormalTok{rad) }\OperatorTok\StringTok{ }
\StringTok{  }\KeywordTok{group_by}\NormalTok{(key, rad) }\OperatorTok
\StringTok{  }\KeywordTok{ggplot}\NormalTok{(}\KeywordTok{aes}\NormalTok{(}\DataTypeTok{x =}\NormalTok{ rad, }\DataTypeTok{y =}\NormalTok{ value, }\DataTypeTok{group =}\NormalTok{ rad)) }\OperatorTok{+}\StringTok{ }
\StringTok{    }\KeywordTok{geom_boxplot}\NormalTok{(}\DataTypeTok{outlier.size =} \FloatTok{0.5}\NormalTok{, }\DataTypeTok{varwidth =}\NormalTok{ T) }\OperatorTok{+}
\StringTok{    }\KeywordTok{facet_wrap}\NormalTok{(}\OperatorTok{~}\StringTok{ }\NormalTok{key, }\DataTypeTok{ncol =} \DecValTok{3}\NormalTok{, }\DataTypeTok{scales =} \StringTok{"free"}\NormalTok{) }\OperatorTok{+}
\StringTok{  }\KeywordTok{ggsave}\NormalTok{(}\StringTok{'plots/rad-boxplot.pdf'}\NormalTok{)}
\end{Highlighting}
\end{Shaded}

\begin{verbatim}
## Saving 6.5 x 4.5 in image
\end{verbatim}

\includegraphics{tidyverse-class_files/figure-latex/boxplot-rad-1.pdf}

Looking at \texttt{lstat} relationships.

\begin{Shaded}
\begin{Highlighting}[]
\NormalTok{boston }\OperatorTok
\StringTok{  }\KeywordTok{keep}\NormalTok{(is.numeric) }\OperatorTok\StringTok{ }
\StringTok{  }\KeywordTok{gather}\NormalTok{( , , }\OperatorTok{-}\NormalTok{lstat) }\OperatorTok\StringTok{ }
\StringTok{  }\KeywordTok{mutate}\NormalTok{(}\DataTypeTok{lstat_gr =} \KeywordTok{ntile}\NormalTok{(lstat, }\DecValTok{10}\NormalTok{)) }\OperatorTok\StringTok{ }
\StringTok{  }\KeywordTok{group_by}\NormalTok{(key, lstat_gr) }\OperatorTok
\StringTok{  }\KeywordTok{ggplot}\NormalTok{(}\KeywordTok{aes}\NormalTok{(}\DataTypeTok{x =}\NormalTok{ lstat_gr, }\DataTypeTok{y =}\NormalTok{ value, }\DataTypeTok{group =}\NormalTok{ lstat_gr)) }\OperatorTok{+}\StringTok{ }
\StringTok{    }\KeywordTok{geom_violin}\NormalTok{() }\OperatorTok{+}
\StringTok{    }\KeywordTok{facet_wrap}\NormalTok{(}\OperatorTok{~}\StringTok{ }\NormalTok{key, }\DataTypeTok{ncol =} \DecValTok{3}\NormalTok{, }\DataTypeTok{scales =} \StringTok{"free"}\NormalTok{) }\OperatorTok{+}
\StringTok{  }\KeywordTok{ggsave}\NormalTok{(}\StringTok{'plots/lstat-violin.pdf'}\NormalTok{)}
\end{Highlighting}
\end{Shaded}

\begin{verbatim}
## Saving 6.5 x 4.5 in image
\end{verbatim}

\includegraphics{tidyverse-class_files/figure-latex/violin-lstat-1.pdf}

Jittering works well for single plots.

\begin{Shaded}
\begin{Highlighting}[]
\NormalTok{boston }\OperatorTok
\StringTok{  }\KeywordTok{ggplot}\NormalTok{(}\KeywordTok{aes}\NormalTok{(tax, nox)) }\OperatorTok{+}
\StringTok{    }\KeywordTok{geom_jitter}\NormalTok{(}\KeywordTok{aes}\NormalTok{(}\DataTypeTok{color =}\NormalTok{ medv, }\DataTypeTok{shape =}\NormalTok{ chas), }
                \DataTypeTok{height =} \FloatTok{0.02}\NormalTok{, }\DataTypeTok{width =} \DecValTok{10}\NormalTok{) }\OperatorTok{+}
\StringTok{    }\KeywordTok{scale_color_gradient2}\NormalTok{(}\DataTypeTok{midpoint =} \DecValTok{20}\NormalTok{, }
                          \DataTypeTok{low =} \StringTok{"blue"}\NormalTok{, }\DataTypeTok{mid =} \StringTok{"gray75"}\NormalTok{, }\DataTypeTok{high =} \StringTok{"red"}\NormalTok{) }\OperatorTok{+}
\StringTok{    }\KeywordTok{geom_hline}\NormalTok{(}\DataTypeTok{yintercept =} \FloatTok{0.6}\NormalTok{, }\DataTypeTok{color =} \StringTok{"yellow"}\NormalTok{) }\OperatorTok{+}
\StringTok{    }\KeywordTok{geom_abline}\NormalTok{(}\DataTypeTok{slope =} \FloatTok{0.001}\NormalTok{, }\DataTypeTok{intercept =} \FloatTok{0.1}\NormalTok{, }\DataTypeTok{color =} \StringTok{"blue"}\NormalTok{, }\DataTypeTok{lty =} \StringTok{"93133313"}\NormalTok{)}
\end{Highlighting}
\end{Shaded}

\includegraphics{tidyverse-class_files/figure-latex/tax-nox-1.pdf}

\begin{Shaded}
\begin{Highlighting}[]
  \KeywordTok{ggsave}\NormalTok{(}\StringTok{'plots/tax-nox.pdf'}\NormalTok{)}
\end{Highlighting}
\end{Shaded}

\begin{verbatim}
## Saving 6.5 x 4.5 in image
\end{verbatim}

\section{Many plotting options}\label{many-plotting-options}

Statistics can be added to the plot as an additional layer. Other layers
are coordinates, facets, and scales.

\begin{Shaded}
\begin{Highlighting}[]
\KeywordTok{ggplot}\NormalTok{(}\DataTypeTok{data =}\NormalTok{ boston) }\OperatorTok{+}\StringTok{ }
\StringTok{  }\KeywordTok{geom_point}\NormalTok{(}\DataTypeTok{mapping =} \KeywordTok{aes}\NormalTok{(}\DataTypeTok{x =}\NormalTok{ rm, }\DataTypeTok{y =}\NormalTok{ medv, }\DataTypeTok{color =}\NormalTok{ crim), }\DataTypeTok{alpha=}\FloatTok{0.75}\NormalTok{) }\OperatorTok{+}
\StringTok{  }\KeywordTok{geom_smooth}\NormalTok{(}\DataTypeTok{mapping =} \KeywordTok{aes}\NormalTok{(}\DataTypeTok{x =}\NormalTok{ rm, }\DataTypeTok{y =}\NormalTok{ medv)) }\OperatorTok{+}
\StringTok{  }\KeywordTok{coord_cartesian}\NormalTok{(}\DataTypeTok{xlim =} \KeywordTok{c}\NormalTok{(}\FloatTok{4.5}\NormalTok{, }\FloatTok{7.5}\NormalTok{)) }\OperatorTok{+}
\StringTok{  }\KeywordTok{scale_y_log10}\NormalTok{() }\OperatorTok{+}
\StringTok{  }\KeywordTok{scale_color_gradient}\NormalTok{(}\DataTypeTok{low =} \StringTok{"yellow"}\NormalTok{, }\DataTypeTok{high =} \StringTok{"blue"}\NormalTok{) }\OperatorTok{+}
\StringTok{  }\KeywordTok{labs}\NormalTok{(}\DataTypeTok{x =} \StringTok{"average rooms / house"}\NormalTok{, }\DataTypeTok{y =} \StringTok{"median house price ($K)"}\NormalTok{,}
       \DataTypeTok{title =} \StringTok{"Boston median house prices vs. average house size"}\NormalTok{)}
\end{Highlighting}
\end{Shaded}

\begin{verbatim}
## `geom_smooth()` using method = 'loess' and formula 'y ~ x'
\end{verbatim}

\includegraphics{tidyverse-class_files/figure-latex/price-rooms-2-1.pdf}

Maybe more useful if colored by quantile of \texttt{crim} value.

\begin{Shaded}
\begin{Highlighting}[]
\NormalTok{boston }\OperatorTok\StringTok{ }
\StringTok{  }\KeywordTok{mutate}\NormalTok{(}\DataTypeTok{crim =} \KeywordTok{cume_dist}\NormalTok{(crim)) }\OperatorTok\StringTok{ }
\StringTok{  }\KeywordTok{ggplot}\NormalTok{() }\OperatorTok{+}
\StringTok{    }\KeywordTok{geom_point}\NormalTok{(}\DataTypeTok{mapping =} \KeywordTok{aes}\NormalTok{(}\DataTypeTok{x =}\NormalTok{ rm, }\DataTypeTok{y =}\NormalTok{ medv, }\DataTypeTok{color =}\NormalTok{ crim), }\DataTypeTok{alpha=}\FloatTok{0.75}\NormalTok{) }\OperatorTok{+}
\StringTok{    }\KeywordTok{geom_smooth}\NormalTok{(}\DataTypeTok{mapping =} \KeywordTok{aes}\NormalTok{(}\DataTypeTok{x =}\NormalTok{ rm, }\DataTypeTok{y =}\NormalTok{ medv)) }\OperatorTok{+}
\StringTok{    }\KeywordTok{coord_cartesian}\NormalTok{(}\DataTypeTok{xlim =} \KeywordTok{c}\NormalTok{(}\FloatTok{4.5}\NormalTok{, }\FloatTok{7.5}\NormalTok{)) }\OperatorTok{+}
\StringTok{    }\KeywordTok{scale_y_log10}\NormalTok{() }\OperatorTok{+}
\StringTok{    }\KeywordTok{scale_color_gradient}\NormalTok{(}\DataTypeTok{low =} \StringTok{"yellow"}\NormalTok{, }\DataTypeTok{high =} \StringTok{"blue"}\NormalTok{) }\OperatorTok{+}
\StringTok{    }\KeywordTok{labs}\NormalTok{(}\DataTypeTok{x =} \StringTok{"average rooms / house"}\NormalTok{, }\DataTypeTok{y =} \StringTok{"median house price ($K)"}\NormalTok{,}
       \DataTypeTok{title =} \StringTok{"Boston median house prices, house size, and crime"}\NormalTok{)}
\end{Highlighting}
\end{Shaded}

\begin{verbatim}
## `geom_smooth()` using method = 'loess' and formula 'y ~ x'
\end{verbatim}

\includegraphics{tidyverse-class_files/figure-latex/price-rooms-3-1.pdf}

Now color by \texttt{rad} but change all 24's to NA's.

\begin{Shaded}
\begin{Highlighting}[]
\NormalTok{boston }\OperatorTok\StringTok{ }
\StringTok{  }\KeywordTok{mutate}\NormalTok{(}\DataTypeTok{rad =} \KeywordTok{ifelse}\NormalTok{(rad }\OperatorTok{==}\StringTok{ }\DecValTok{24}\NormalTok{, }\OtherTok{NA}\NormalTok{, rad)) }\OperatorTok\StringTok{ }
\StringTok{  }\KeywordTok{ggplot}\NormalTok{() }\OperatorTok{+}
\StringTok{    }\KeywordTok{geom_point}\NormalTok{(}\DataTypeTok{mapping =} \KeywordTok{aes}\NormalTok{(}\DataTypeTok{x =}\NormalTok{ rm, }\DataTypeTok{y =}\NormalTok{ medv, }\DataTypeTok{color =}\NormalTok{ rad), }\DataTypeTok{alpha=}\FloatTok{0.75}\NormalTok{) }\OperatorTok{+}
\StringTok{    }\KeywordTok{geom_smooth}\NormalTok{(}\DataTypeTok{mapping =} \KeywordTok{aes}\NormalTok{(}\DataTypeTok{x =}\NormalTok{ rm, }\DataTypeTok{y =}\NormalTok{ medv)) }\OperatorTok{+}
\StringTok{    }\KeywordTok{coord_cartesian}\NormalTok{(}\DataTypeTok{xlim =} \KeywordTok{c}\NormalTok{(}\FloatTok{4.5}\NormalTok{, }\FloatTok{7.5}\NormalTok{)) }\OperatorTok{+}
\StringTok{    }\KeywordTok{scale_y_log10}\NormalTok{() }\OperatorTok{+}
\StringTok{    }\KeywordTok{scale_color_gradient}\NormalTok{(}\DataTypeTok{low =} \StringTok{"yellow"}\NormalTok{, }\DataTypeTok{high =} \StringTok{"red"}\NormalTok{, }\DataTypeTok{na.value =} \StringTok{"black"}\NormalTok{) }\OperatorTok{+}
\StringTok{    }\KeywordTok{labs}\NormalTok{(}\DataTypeTok{x =} \StringTok{"average rooms / house"}\NormalTok{, }\DataTypeTok{y =} \StringTok{"median house price ($K)"}\NormalTok{,}
         \DataTypeTok{title =} \StringTok{"Boston median house prices and access to radial highways"}\NormalTok{)}
\end{Highlighting}
\end{Shaded}

\begin{verbatim}
## `geom_smooth()` using method = 'loess' and formula 'y ~ x'
\end{verbatim}

\includegraphics{tidyverse-class_files/figure-latex/price-rad-na-1.pdf}

Maybe excluding newly-NA'ed \texttt{rad} values helps the crime plot.

\begin{Shaded}
\begin{Highlighting}[]
\NormalTok{boston }\OperatorTok\StringTok{ }
\StringTok{  }\KeywordTok{filter}\NormalTok{(}\OperatorTok{!}\NormalTok{rad }\OperatorTok{==}\StringTok{ }\DecValTok{24}\NormalTok{) }\OperatorTok\StringTok{ }
\StringTok{  }\KeywordTok{mutate}\NormalTok{(}\DataTypeTok{crim =} \KeywordTok{cume_dist}\NormalTok{(crim)) }\OperatorTok\StringTok{ }
\StringTok{  }\KeywordTok{ggplot}\NormalTok{() }\OperatorTok{+}
\StringTok{    }\KeywordTok{geom_point}\NormalTok{(}\DataTypeTok{mapping =} \KeywordTok{aes}\NormalTok{(}\DataTypeTok{x =}\NormalTok{ rm, }\DataTypeTok{y =}\NormalTok{ medv, }\DataTypeTok{color =}\NormalTok{ crim), }\DataTypeTok{size =} \DecValTok{1}\NormalTok{) }\OperatorTok{+}
\StringTok{    }\KeywordTok{geom_smooth}\NormalTok{(}\DataTypeTok{mapping =} \KeywordTok{aes}\NormalTok{(}\DataTypeTok{x =}\NormalTok{ rm, }\DataTypeTok{y =}\NormalTok{ medv), }\DataTypeTok{lwd =} \FloatTok{0.5}\NormalTok{) }\OperatorTok{+}
\StringTok{    }\KeywordTok{scale_y_log10}\NormalTok{() }\OperatorTok{+}
\StringTok{    }\KeywordTok{scale_color_gradient}\NormalTok{(}\DataTypeTok{low =} \StringTok{"yellow"}\NormalTok{, }\DataTypeTok{high =} \StringTok{"blue"}\NormalTok{) }\OperatorTok{+}
\StringTok{    }\KeywordTok{labs}\NormalTok{(}\DataTypeTok{x =} \StringTok{"average rooms / house"}\NormalTok{, }\DataTypeTok{y =} \StringTok{"median house price ($K)"}\NormalTok{,}
       \DataTypeTok{title =} \StringTok{"Boston median house prices, house size, and crime"}\NormalTok{)}
\end{Highlighting}
\end{Shaded}

\begin{verbatim}
## `geom_smooth()` using method = 'loess' and formula 'y ~ x'
\end{verbatim}

\includegraphics{tidyverse-class_files/figure-latex/price-rooms-4-1.pdf}

A grid of \texttt{nox} vs. \texttt{dis} plots according to \texttt{chas}
(rows) and binned level (ntile) of \texttt{rad}.

\begin{Shaded}
\begin{Highlighting}[]
\NormalTok{boston }\OperatorTok
\StringTok{  }\KeywordTok{mutate}\NormalTok{(}\DataTypeTok{rad =} \KeywordTok{ifelse}\NormalTok{(rad }\OperatorTok{==}\StringTok{ }\DecValTok{24}\NormalTok{, }\OtherTok{NA}\NormalTok{, rad)) }\OperatorTok\StringTok{ }
\StringTok{  }\KeywordTok{filter}\NormalTok{(}\OperatorTok{!}\KeywordTok{is.na}\NormalTok{(rad)) }\OperatorTok\StringTok{ }
\StringTok{  }\KeywordTok{ggplot}\NormalTok{(}\KeywordTok{aes}\NormalTok{(nox, dis, }\DataTypeTok{color =}\NormalTok{ medv)) }\OperatorTok{+}
\StringTok{    }\KeywordTok{geom_jitter}\NormalTok{() }\OperatorTok{+}
\StringTok{    }\KeywordTok{facet_grid}\NormalTok{(chas }\OperatorTok{~}\StringTok{ }\KeywordTok{ntile}\NormalTok{(rad, }\DecValTok{3}\NormalTok{)) }\OperatorTok{+}
\StringTok{    }\KeywordTok{geom_smooth}\NormalTok{()}
\end{Highlighting}
\end{Shaded}

\begin{verbatim}
## `geom_smooth()` using method = 'loess' and formula 'y ~ x'
\end{verbatim}

\includegraphics{tidyverse-class_files/figure-latex/facet-grid-1.pdf}

Multiplots available with \texttt{gridExtra}, used by ggplot2.

\begin{Shaded}
\begin{Highlighting}[]
\KeywordTok{require}\NormalTok{(gridExtra)}
\NormalTok{p1 <-}\StringTok{ }\KeywordTok{ggplot}\NormalTok{(boston) }\OperatorTok{+}
\StringTok{  }\KeywordTok{geom_point}\NormalTok{(}\KeywordTok{aes}\NormalTok{(nox, medv, }\DataTypeTok{shape =}\NormalTok{ chas, }\DataTypeTok{alpha =} \KeywordTok{cume_dist}\NormalTok{(lstat)), }
             \DataTypeTok{color =} \StringTok{'red'}\NormalTok{, }\DataTypeTok{size =} \DecValTok{2}\NormalTok{) }\OperatorTok{+}
\StringTok{  }\KeywordTok{labs}\NormalTok{(}\DataTypeTok{title =} \StringTok{'using aes(alpha, shape)'}\NormalTok{)}
\NormalTok{p2 <-}\StringTok{ }\NormalTok{boston }\OperatorTok
\StringTok{  }\KeywordTok{mutate}\NormalTok{(}\DataTypeTok{lstat_gr =} \KeywordTok{ntile}\NormalTok{(lstat, }\DecValTok{3}\NormalTok{)) }\OperatorTok\StringTok{ }
\StringTok{  }\KeywordTok{ggplot}\NormalTok{(}\KeywordTok{aes}\NormalTok{(tax, medv, }\DataTypeTok{color =}\NormalTok{ lstat_gr, }\DataTypeTok{size =}\NormalTok{ nox)) }\OperatorTok{+}\StringTok{ }
\StringTok{  }\KeywordTok{geom_point}\NormalTok{(}\DataTypeTok{shape =} \DecValTok{16}\NormalTok{, }\DataTypeTok{alpha =} \FloatTok{0.75}\NormalTok{) }\OperatorTok{+}
\StringTok{  }\KeywordTok{geom_smooth}\NormalTok{(}\KeywordTok{aes}\NormalTok{(}\DataTypeTok{group =}\NormalTok{ lstat_gr), }\DataTypeTok{lwd =} \FloatTok{0.8}\NormalTok{) }\OperatorTok{+}
\StringTok{  }\KeywordTok{labs}\NormalTok{(}\DataTypeTok{title =} \StringTok{'using aes(group, size)'}\NormalTok{)}
\NormalTok{p1}
\end{Highlighting}
\end{Shaded}

\includegraphics{tidyverse-class_files/figure-latex/multiple-plots-1.pdf}

\begin{Shaded}
\begin{Highlighting}[]
\NormalTok{p2}
\end{Highlighting}
\end{Shaded}

\begin{verbatim}
## `geom_smooth()` using method = 'loess' and formula 'y ~ x'
\end{verbatim}

\includegraphics{tidyverse-class_files/figure-latex/multiple-plots-2.pdf}

\begin{Shaded}
\begin{Highlighting}[]
\KeywordTok{ggsave}\NormalTok{(}\StringTok{'plots/two-plot.pdf'}\NormalTok{, }\KeywordTok{arrangeGrob}\NormalTok{(p1, p2))}
\end{Highlighting}
\end{Shaded}

\begin{verbatim}
## Saving 6.5 x 4.5 in image
## `geom_smooth()` using method = 'loess' and formula 'y ~ x'
\end{verbatim}

\chapter{dplyr and tidyr}\label{ch:dplyr}

\begin{Shaded}
\begin{Highlighting}[]
\KeywordTok{library}\NormalTok{(tidyverse)}
\KeywordTok{library}\NormalTok{(gridExtra)}
\NormalTok{batting <-}\StringTok{ }\KeywordTok{as_tibble}\NormalTok{(Lahman}\OperatorTok{::}\NormalTok{Batting)}
\NormalTok{fielding <-}\StringTok{ }\KeywordTok{as_tibble}\NormalTok{(Lahman}\OperatorTok{::}\NormalTok{Fielding)}
\end{Highlighting}
\end{Shaded}

\section{Hoofin' it with dplyr}\label{hoofin-it-with-dplyr}

Condense batting stats into player career totals, keep only those
\textgreater{}= 500 games.

\begin{Shaded}
\begin{Highlighting}[]
\NormalTok{is_col <-}\StringTok{ }\KeywordTok{names}\NormalTok{(batting)[}\KeywordTok{c}\NormalTok{(}\DecValTok{1}\NormalTok{, }\DecValTok{2}\NormalTok{, }\DecValTok{4}\NormalTok{, }\DecValTok{6}\OperatorTok{:}\DecValTok{17}\NormalTok{)]}
\NormalTok{is_num <-}\StringTok{ }\KeywordTok{names}\NormalTok{(batting)[}\KeywordTok{sapply}\NormalTok{(batting, is.numeric)]}
\NormalTok{gt_}\DecValTok{500}\NormalTok{ <-}\StringTok{ }\NormalTok{batting }\OperatorTok
\StringTok{  }\KeywordTok{select}\NormalTok{(is_col) }\OperatorTok
\StringTok{  }\KeywordTok{select}\NormalTok{(}\OperatorTok{-}\NormalTok{teamID) }\OperatorTok\StringTok{ }
\StringTok{  }\KeywordTok{drop_na}\NormalTok{() }\OperatorTok
\StringTok{  }\KeywordTok{group_by}\NormalTok{(playerID) }\OperatorTok\StringTok{ }
\StringTok{  }\KeywordTok{summarize_at}\NormalTok{(is_col[}\OperatorTok{-}\NormalTok{(}\DecValTok{1}\OperatorTok{:}\DecValTok{3}\NormalTok{)], sum, }\DataTypeTok{na.rm =}\NormalTok{ T) }\OperatorTok\StringTok{ }
\StringTok{  }\KeywordTok{filter}\NormalTok{(G }\OperatorTok{>=}\StringTok{ }\DecValTok{500}\NormalTok{)}
\end{Highlighting}
\end{Shaded}

All Ha\textasciitilde{} Green\textasciitilde{} statistics to confirm
that the data reduction looks right:

\begin{Shaded}
\begin{Highlighting}[]
\NormalTok{batting }\OperatorTok
\StringTok{  }\KeywordTok{filter}\NormalTok{(}\KeywordTok{str_detect}\NormalTok{(playerID, }\StringTok{"greenha"}\NormalTok{))  }\CommentTok{# a taste of `stringr`}
\end{Highlighting}
\end{Shaded}

\begin{verbatim}
## # A tibble: 14 x 22
##    playerID yearID stint teamID lgID      G    AB     R     H   X2B   X3B
##    <chr>     <int> <int> <fct>  <fct> <int> <int> <int> <int> <int> <int>
##  1 greenha~   1930     1 DET    AL        1     1     0     0     0     0
##  2 greenha~   1933     1 DET    AL      117   449    59   135    33     3
##  3 greenha~   1934     1 DET    AL      153   593   118   201    63     7
##  4 greenha~   1935     1 DET    AL      152   619   121   203    46    16
##  5 greenha~   1935     1 BRO    NL        2     0     0     0     0     0
##  6 greenha~   1936     1 DET    AL       12    46    10    16     6     2
##  7 greenha~   1937     1 DET    AL      154   594   137   200    49    14
##  8 greenha~   1938     1 DET    AL      155   556   144   175    23     4
##  9 greenha~   1939     1 DET    AL      138   500   112   156    42     7
## 10 greenha~   1940     1 DET    AL      148   573   129   195    50     8
## 11 greenha~   1941     1 DET    AL       19    67    12    18     5     1
## 12 greenha~   1945     1 DET    AL       78   270    47    84    20     2
## 13 greenha~   1946     1 DET    AL      142   523    91   145    29     5
## 14 greenha~   1947     1 PIT    NL      125   402    71   100    13     2
## # ... with 11 more variables: HR <int>, RBI <int>, SB <int>, CS <int>,
## #   BB <int>, SO <int>, IBB <int>, HBP <int>, SH <int>, SF <int>,
## #   GIDP <int>
\end{verbatim}

Positions by game.

\begin{Shaded}
\begin{Highlighting}[]
\NormalTok{fielding }\OperatorTok\StringTok{ }
\StringTok{  }\KeywordTok{group_by}\NormalTok{(POS) }\OperatorTok\StringTok{ }
\StringTok{  }\KeywordTok{count}\NormalTok{(}\DataTypeTok{wt =}\NormalTok{ G)}
\end{Highlighting}
\end{Shaded}

\begin{verbatim}
## # A tibble: 7 x 2
## # Groups:   POS [7]
##   POS         n
##   <chr>   <int>
## 1 1B     482698
## 2 2B     480968
## 3 3B     482320
## 4 C      497547
## 5 OF    1451301
## 6 P     1106574
## 7 SS     479045
\end{verbatim}

Attach a column denoting their main fielding position.

\begin{Shaded}
\begin{Highlighting}[]
\NormalTok{is_field =}\StringTok{ }\KeywordTok{names}\NormalTok{(fielding)[}\KeywordTok{c}\NormalTok{(}\DecValTok{1}\NormalTok{, }\DecValTok{6}\NormalTok{, }\DecValTok{7}\NormalTok{, }\DecValTok{9}\NormalTok{, }\DecValTok{10}\NormalTok{, }\DecValTok{11}\NormalTok{, }\DecValTok{12}\NormalTok{, }\DecValTok{13}\NormalTok{)]}
\NormalTok{fielding }\OperatorTok
\StringTok{  }\KeywordTok{select}\NormalTok{(is_field) }\OperatorTok\StringTok{ }
\StringTok{  }\KeywordTok{map}\NormalTok{(}\OperatorTok{~}\StringTok{ }\KeywordTok{sum}\NormalTok{(}\KeywordTok{is.na}\NormalTok{(.)))}
\end{Highlighting}
\end{Shaded}

\begin{verbatim}
## $playerID
## [1] 0
## 
## $POS
## [1] 0
## 
## $G
## [1] 0
## 
## $InnOuts
## [1] 29929
## 
## $PO
## [1] 0
## 
## $A
## [1] 0
## 
## $E
## [1] 1
## 
## $DP
## [1] 0
\end{verbatim}

That's odd, just one error NA.

\begin{Shaded}
\begin{Highlighting}[]
\NormalTok{fielding }\OperatorTok\StringTok{ }
\StringTok{  }\KeywordTok{filter}\NormalTok{(}\KeywordTok{is.na}\NormalTok{(E))}
\end{Highlighting}
\end{Shaded}

\begin{verbatim}
## # A tibble: 1 x 18
##   playerID yearID stint teamID lgID  POS       G    GS InnOuts    PO     A
##   <chr>     <int> <int> <fct>  <fct> <chr> <int> <int>   <int> <int> <int>
## 1 fordbi01   1936     1 BSN    NL    P         1    NA      NA     0     0
## # ... with 7 more variables: E <int>, DP <int>, PB <int>, WP <int>,
## #   SB <int>, CS <int>, ZR <int>
\end{verbatim}

Removing InnOuts is a good idea, too many missing, and those
\texttt{NA}s aren't relevant to the analysis.

\begin{Shaded}
\begin{Highlighting}[]
\NormalTok{is_field =}\StringTok{ }\KeywordTok{names}\NormalTok{(fielding)[}\KeywordTok{c}\NormalTok{(}\DecValTok{1}\NormalTok{, }\DecValTok{6}\NormalTok{, }\DecValTok{7}\NormalTok{, }\DecValTok{10}\NormalTok{, }\DecValTok{11}\NormalTok{, }\DecValTok{12}\NormalTok{, }\DecValTok{13}\NormalTok{)]}
\NormalTok{pos_tot <-}\StringTok{ }\NormalTok{fielding }\OperatorTok
\StringTok{  }\KeywordTok{select}\NormalTok{(is_field) }\OperatorTok\StringTok{   }\CommentTok{# cull columns}
\StringTok{  }\KeywordTok{drop_na}\NormalTok{() }\OperatorTok\StringTok{  }\CommentTok{# drop the missing value}
\StringTok{  }\KeywordTok{group_by}\NormalTok{(playerID, POS) }\OperatorTok\StringTok{  }\CommentTok{# want the most G by POS assigned to playerID}
\StringTok{  }\KeywordTok{summarize_all}\NormalTok{(sum) }\OperatorTok
\StringTok{  }\KeywordTok{ungroup}\NormalTok{() }\OperatorTok\StringTok{ }
\StringTok{  }\KeywordTok{filter}\NormalTok{(G }\OperatorTok{>=}\StringTok{ }\DecValTok{100}\NormalTok{) }\OperatorTok\StringTok{  }\CommentTok{# only those with 100 G at a POS}
\StringTok{  }\KeywordTok{arrange}\NormalTok{(playerID, }\KeywordTok{desc}\NormalTok{(G)) }\OperatorTok\StringTok{  }\CommentTok{# if G instead of desc(G), use last(POS)}
\StringTok{  }\KeywordTok{group_by}\NormalTok{(playerID) }\OperatorTok\StringTok{ }
\StringTok{  }\KeywordTok{mutate}\NormalTok{(}\DataTypeTok{pos1 =} \KeywordTok{first}\NormalTok{(POS)) }\OperatorTok\StringTok{ }
\StringTok{  }\KeywordTok{filter}\NormalTok{(POS }\OperatorTok{==}\StringTok{ }\NormalTok{pos1) }\OperatorTok\StringTok{  }\CommentTok{# assign position with most games to POS}
\StringTok{  }\KeywordTok{select}\NormalTok{(}\OperatorTok{-}\NormalTok{pos1)}
\end{Highlighting}
\end{Shaded}

\section{tidyr and relational data}\label{tidyr-and-relational-data}

Add fielding info to batting tibble.

\begin{Shaded}
\begin{Highlighting}[]
\NormalTok{(batpos <-}\StringTok{ }\NormalTok{gt_}\DecValTok{500} \OperatorTok\StringTok{ }
\StringTok{   }\KeywordTok{left_join}\NormalTok{(pos_tot, }\DataTypeTok{by =} \StringTok{"playerID"}\NormalTok{, }\DataTypeTok{suffix =} \KeywordTok{c}\NormalTok{(}\StringTok{".h"}\NormalTok{, }\StringTok{".f"}\NormalTok{)))}
\end{Highlighting}
\end{Shaded}

\begin{verbatim}
## # A tibble: 2,667 x 19
##    playerID   G.h    AB     R     H   X2B   X3B    HR   RBI    SB    CS
##    <chr>    <int> <int> <int> <int> <int> <int> <int> <int> <int> <int>
##  1 aaronha~  3298 12364  2174  3771   624    98   755  2297   240    73
##  2 abbotku~   702  2044   273   523   109    23    62   242    22    11
##  3 abernte~   681   181    12    25     3     0     0     9     0     0
##  4 abramca~   521  1543   246   422    62    19    32   134    11    18
##  5 abreubo~  2425  8480  1453  2470   574    59   288  1363   400   128
##  6 abreujo~   742  2913   398   858   180    13   146   488     8     3
##  7 ackledu~   635  2125   261   512    94    18    46   216    31    12
##  8 adairje~  1165  4019   378  1022   163    19    57   366    29    29
##  9 adamsbo~   797  2604   395   701   107    31    25   188    25    30
## 10 adamsgl~   661  1617   152   452    79     5    34   225     6    10
## # ... with 2,657 more rows, and 8 more variables: BB <int>, SO <int>,
## #   POS <chr>, G.f <int>, PO <int>, A <int>, E <int>, DP <int>
\end{verbatim}

Counts of positions.

\begin{Shaded}
\begin{Highlighting}[]
\NormalTok{batpos }\OperatorTok\StringTok{ }
\StringTok{  }\KeywordTok{group_by}\NormalTok{(POS) }\OperatorTok\StringTok{ }
\StringTok{  }\KeywordTok{count}\NormalTok{()}
\end{Highlighting}
\end{Shaded}

\begin{verbatim}
## # A tibble: 8 x 2
## # Groups:   POS [8]
##   POS       n
##   <chr> <int>
## 1 <NA>      2
## 2 1B      254
## 3 2B      277
## 4 3B      270
## 5 C       300
## 6 OF      890
## 7 P       378
## 8 SS      296
\end{verbatim}

NAs are likely DHs.

\begin{Shaded}
\begin{Highlighting}[]
\NormalTok{pos_nas <-}\StringTok{ }\NormalTok{batpos }\OperatorTok\StringTok{ }
\StringTok{  }\KeywordTok{filter}\NormalTok{(}\KeywordTok{is.na}\NormalTok{(POS))}
\NormalTok{batting }\OperatorTok\StringTok{ }
\StringTok{  }\KeywordTok{inner_join}\NormalTok{(pos_nas, }\DataTypeTok{by =} \StringTok{"playerID"}\NormalTok{)}
\end{Highlighting}
\end{Shaded}

\begin{verbatim}
## # A tibble: 26 x 40
##    playerID yearID stint teamID lgID      G  AB.x   R.x   H.x X2B.x X3B.x
##    <chr>     <int> <int> <fct>  <fct> <int> <int> <int> <int> <int> <int>
##  1 moraljo~   1973     1 OAK    AL        6    14     0     4     1     0
##  2 moraljo~   1973     2 MON    NL        5     5     0     2     0     0
##  3 moraljo~   1974     1 MON    NL       25    26     3     7     4     0
##  4 moraljo~   1975     1 MON    NL       93   163    18    49     6     1
##  5 moraljo~   1976     1 MON    NL      104   158    12    50    11     0
##  6 moraljo~   1977     1 MON    NL       65    74     3    15     4     1
##  7 moraljo~   1978     1 MIN    AL      101   242    22    76    13     1
##  8 moraljo~   1979     1 MIN    AL       92   191    21    51     5     1
##  9 moraljo~   1980     1 MIN    AL       97   241    36    73    17     2
## 10 moraljo~   1981     1 BAL    AL       38    86     6    21     3     0
## # ... with 16 more rows, and 29 more variables: HR.x <int>, RBI.x <int>,
## #   SB.x <int>, CS.x <int>, BB.x <int>, SO.x <int>, IBB <int>, HBP <int>,
## #   SH <int>, SF <int>, GIDP <int>, G.h <int>, AB.y <int>, R.y <int>,
## #   H.y <int>, X2B.y <int>, X3B.y <int>, HR.y <int>, RBI.y <int>,
## #   SB.y <int>, CS.y <int>, BB.y <int>, SO.y <int>, POS <chr>, G.f <int>,
## #   PO <int>, A <int>, E <int>, DP <int>
\end{verbatim}

Drop these two DHs.

\begin{Shaded}
\begin{Highlighting}[]
\NormalTok{batpos <-}\StringTok{ }\NormalTok{batpos }\OperatorTok\StringTok{ }
\StringTok{  }\KeywordTok{drop_na}\NormalTok{()}
\end{Highlighting}
\end{Shaded}

Now we could explore many aspects of hitting stats vs.~position, and see
what position players were better fielders or better hitters, or if
neither we can see if they played for the Expos.

\begin{Shaded}
\begin{Highlighting}[]
\NormalTok{batpos }\OperatorTok\StringTok{ }
\StringTok{  }\KeywordTok{filter}\NormalTok{(POS }\OperatorTok{==}\StringTok{ "SS"}\NormalTok{) }\OperatorTok\StringTok{ }
\StringTok{  }\KeywordTok{mutate}\NormalTok{(}\DataTypeTok{BA =}\NormalTok{ H }\OperatorTok{/}\StringTok{ }\NormalTok{AB) }\OperatorTok\StringTok{  }\CommentTok{# batting average, hits / at bats}
\StringTok{  }\KeywordTok{mutate}\NormalTok{(}\DataTypeTok{Err =}\NormalTok{ E }\OperatorTok{/}\StringTok{ }\NormalTok{(PO }\OperatorTok{+}\StringTok{ }\NormalTok{A)) }\OperatorTok\StringTok{   }\CommentTok{# error rate, errors / (put outs + assists)}
\StringTok{  }\KeywordTok{mutate}\NormalTok{(}\DataTypeTok{HRR =}\NormalTok{ HR }\OperatorTok{/}\StringTok{ }\NormalTok{AB) }\OperatorTok\StringTok{  }\CommentTok{# home run rate, home runs / at bats}
\StringTok{  }\KeywordTok{ggplot}\NormalTok{(}\KeywordTok{aes}\NormalTok{(Err, BA)) }\OperatorTok{+}
\StringTok{    }\KeywordTok{geom_point}\NormalTok{(}\KeywordTok{aes}\NormalTok{(}\DataTypeTok{color =}\NormalTok{ HRR)) }\OperatorTok{+}
\StringTok{    }\KeywordTok{geom_smooth}\NormalTok{()}
\end{Highlighting}
\end{Shaded}

\begin{verbatim}
## `geom_smooth()` using method = 'loess' and formula 'y ~ x'
\end{verbatim}

\includegraphics{tidyverse-class_files/figure-latex/ss-stats-1.pdf}

\begin{Shaded}
\begin{Highlighting}[]
\NormalTok{temp <-}\StringTok{ }\NormalTok{batpos }\OperatorTok\StringTok{ }
\StringTok{  }\KeywordTok{mutate}\NormalTok{(}\DataTypeTok{BA =}\NormalTok{ H }\OperatorTok{/}\StringTok{ }\NormalTok{AB) }\OperatorTok\StringTok{  }\CommentTok{# batting average, hits / at bats}
\StringTok{  }\KeywordTok{filter}\NormalTok{(}\KeywordTok{between}\NormalTok{(BA, }\FloatTok{0.01}\NormalTok{, }\FloatTok{0.49}\NormalTok{)) }\OperatorTok\StringTok{ }
\StringTok{  }\KeywordTok{mutate}\NormalTok{(}\DataTypeTok{Err =}\NormalTok{ E }\OperatorTok{/}\StringTok{ }\NormalTok{(PO }\OperatorTok{+}\StringTok{ }\NormalTok{A)) }\OperatorTok\StringTok{   }\CommentTok{# error rate, errors / (put outs + assists)}
\StringTok{  }\KeywordTok{mutate}\NormalTok{(}\DataTypeTok{HRR =}\NormalTok{ HR }\OperatorTok{/}\StringTok{ }\NormalTok{AB)  }\CommentTok{# home run rate, home runs / at bats}
\NormalTok{p1 <-}\StringTok{ }\NormalTok{temp }\OperatorTok\StringTok{ }
\StringTok{  }\KeywordTok{ggplot}\NormalTok{(}\KeywordTok{aes}\NormalTok{(Err, BA, }\DataTypeTok{color =}\NormalTok{ POS)) }\OperatorTok{+}
\StringTok{    }\KeywordTok{geom_point}\NormalTok{(}\DataTypeTok{alpha =} \FloatTok{0.5}\NormalTok{, }\DataTypeTok{size =} \FloatTok{0.5}\NormalTok{) }\OperatorTok{+}
\StringTok{    }\KeywordTok{geom_smooth}\NormalTok{(}\KeywordTok{aes}\NormalTok{(}\DataTypeTok{group =}\NormalTok{ POS)) }\OperatorTok{+}
\StringTok{    }\KeywordTok{coord_cartesian}\NormalTok{(}\DataTypeTok{xlim =} \KeywordTok{c}\NormalTok{(}\DecValTok{0}\NormalTok{, }\FloatTok{0.1}\NormalTok{), }\DataTypeTok{ylim =} \KeywordTok{c}\NormalTok{(}\FloatTok{0.1}\NormalTok{, }\FloatTok{0.42}\NormalTok{))}
\NormalTok{p2 <-}\StringTok{ }\NormalTok{temp }\OperatorTok\StringTok{ }
\StringTok{  }\KeywordTok{filter}\NormalTok{(POS }\OperatorTok{!=}\StringTok{ "P"}\NormalTok{) }\OperatorTok\StringTok{ }
\StringTok{  }\KeywordTok{ggplot}\NormalTok{(}\KeywordTok{aes}\NormalTok{(BA, HRR, }\DataTypeTok{color =}\NormalTok{ POS)) }\OperatorTok{+}
\StringTok{    }\KeywordTok{geom_point}\NormalTok{(}\DataTypeTok{alpha =} \FloatTok{0.5}\NormalTok{, }\DataTypeTok{size =} \FloatTok{0.5}\NormalTok{) }\OperatorTok{+}
\StringTok{    }\KeywordTok{geom_smooth}\NormalTok{(}\KeywordTok{aes}\NormalTok{(}\DataTypeTok{group =}\NormalTok{ POS))}
\NormalTok{p1}
\end{Highlighting}
\end{Shaded}

\begin{verbatim}
## `geom_smooth()` using method = 'loess' and formula 'y ~ x'
\end{verbatim}

\includegraphics{tidyverse-class_files/figure-latex/plot-pos-ba-1.pdf}

\begin{Shaded}
\begin{Highlighting}[]
\NormalTok{p2}
\end{Highlighting}
\end{Shaded}

\begin{verbatim}
## `geom_smooth()` using method = 'loess' and formula 'y ~ x'
\end{verbatim}

\includegraphics{tidyverse-class_files/figure-latex/plot-pos-ba-2.pdf}

\begin{Shaded}
\begin{Highlighting}[]
\KeywordTok{ggsave}\NormalTok{(}\StringTok{'plots/pos-bat.pdf'}\NormalTok{, }\KeywordTok{arrangeGrob}\NormalTok{(p1, p2))}
\end{Highlighting}
\end{Shaded}

\begin{verbatim}
## Saving 6.5 x 4.5 in image
## `geom_smooth()` using method = 'loess' and formula 'y ~ x'
## `geom_smooth()` using method = 'loess' and formula 'y ~ x'
\end{verbatim}

\chapter{dplyr closures and rlang}\label{ch:rlang}

\begin{Shaded}
\begin{Highlighting}[]
\KeywordTok{library}\NormalTok{(tidyverse)}
\KeywordTok{library}\NormalTok{(gridExtra)}
\NormalTok{batting <-}\StringTok{ }\KeywordTok{as_tibble}\NormalTok{(Lahman}\OperatorTok{::}\NormalTok{Batting)}
\NormalTok{fielding <-}\StringTok{ }\KeywordTok{as_tibble}\NormalTok{(Lahman}\OperatorTok{::}\NormalTok{Fielding)}

\NormalTok{is_col <-}\StringTok{ }\KeywordTok{names}\NormalTok{(batting)[}\KeywordTok{c}\NormalTok{(}\DecValTok{1}\NormalTok{, }\DecValTok{2}\NormalTok{, }\DecValTok{4}\NormalTok{, }\DecValTok{6}\OperatorTok{:}\DecValTok{17}\NormalTok{)]}
\NormalTok{is_num <-}\StringTok{ }\KeywordTok{names}\NormalTok{(batting)[}\KeywordTok{sapply}\NormalTok{(batting, is.numeric)]}
\NormalTok{gt_}\DecValTok{500}\NormalTok{ <-}\StringTok{ }\NormalTok{batting }\OperatorTok
\StringTok{  }\KeywordTok{select}\NormalTok{(is_col) }\OperatorTok
\StringTok{  }\KeywordTok{select}\NormalTok{(}\OperatorTok{-}\NormalTok{teamID) }\OperatorTok\StringTok{ }
\StringTok{  }\KeywordTok{drop_na}\NormalTok{() }\OperatorTok
\StringTok{  }\KeywordTok{group_by}\NormalTok{(playerID) }\OperatorTok\StringTok{ }
\StringTok{  }\KeywordTok{summarize_at}\NormalTok{(is_col[}\OperatorTok{-}\NormalTok{(}\DecValTok{1}\OperatorTok{:}\DecValTok{3}\NormalTok{)], sum, }\DataTypeTok{na.rm =}\NormalTok{ T) }\OperatorTok\StringTok{ }
\StringTok{  }\KeywordTok{filter}\NormalTok{(G }\OperatorTok{>=}\StringTok{ }\DecValTok{500}\NormalTok{)}
\NormalTok{is_field =}\StringTok{ }\KeywordTok{names}\NormalTok{(fielding)[}\KeywordTok{c}\NormalTok{(}\DecValTok{1}\NormalTok{, }\DecValTok{6}\NormalTok{, }\DecValTok{7}\NormalTok{, }\DecValTok{10}\NormalTok{, }\DecValTok{11}\NormalTok{, }\DecValTok{12}\NormalTok{, }\DecValTok{13}\NormalTok{)]}
\NormalTok{pos_tot <-}\StringTok{ }\NormalTok{fielding }\OperatorTok
\StringTok{  }\KeywordTok{select}\NormalTok{(is_field) }\OperatorTok\StringTok{ }
\StringTok{  }\KeywordTok{drop_na}\NormalTok{() }\OperatorTok\StringTok{ }
\StringTok{  }\KeywordTok{group_by}\NormalTok{(playerID, POS) }\OperatorTok
\StringTok{  }\KeywordTok{summarize_all}\NormalTok{(sum) }\OperatorTok
\StringTok{  }\KeywordTok{ungroup}\NormalTok{() }\OperatorTok\StringTok{ }
\StringTok{  }\KeywordTok{filter}\NormalTok{(G }\OperatorTok{>=}\StringTok{ }\DecValTok{100}\NormalTok{) }\OperatorTok\StringTok{ }
\StringTok{  }\KeywordTok{arrange}\NormalTok{(playerID, }\KeywordTok{desc}\NormalTok{(G)) }\OperatorTok
\StringTok{  }\KeywordTok{group_by}\NormalTok{(playerID) }\OperatorTok\StringTok{ }
\StringTok{  }\KeywordTok{mutate}\NormalTok{(}\DataTypeTok{pos1 =} \KeywordTok{first}\NormalTok{(POS)) }\OperatorTok\StringTok{ }
\StringTok{  }\KeywordTok{filter}\NormalTok{(POS }\OperatorTok{==}\StringTok{ }\NormalTok{pos1) }\OperatorTok\StringTok{ }
\StringTok{  }\KeywordTok{select}\NormalTok{(}\OperatorTok{-}\NormalTok{pos1)}
\NormalTok{batpos <-}\StringTok{ }\NormalTok{gt_}\DecValTok{500} \OperatorTok\StringTok{ }
\StringTok{  }\KeywordTok{left_join}\NormalTok{(pos_tot, }\DataTypeTok{by =} \StringTok{"playerID"}\NormalTok{)}
\NormalTok{batpos <-}\StringTok{ }\NormalTok{batpos }\OperatorTok\StringTok{ }
\StringTok{  }\KeywordTok{drop_na}\NormalTok{()}
\NormalTok{batpos <-}\StringTok{ }\NormalTok{batpos }\OperatorTok\StringTok{ }
\StringTok{  }\KeywordTok{mutate}\NormalTok{(}\DataTypeTok{BA =}\NormalTok{ H }\OperatorTok{/}\StringTok{ }\NormalTok{AB) }\OperatorTok\StringTok{  }\CommentTok{# batting average, hits / at bats}
\StringTok{  }\KeywordTok{mutate}\NormalTok{(}\DataTypeTok{Err =}\NormalTok{ E }\OperatorTok{/}\StringTok{ }\NormalTok{(PO }\OperatorTok{+}\StringTok{ }\NormalTok{A)) }\OperatorTok\StringTok{   }\CommentTok{# error rate, errors / (put outs + assists)}
\StringTok{  }\KeywordTok{mutate}\NormalTok{(}\DataTypeTok{HRR =}\NormalTok{ HR }\OperatorTok{/}\StringTok{ }\NormalTok{AB)  }\CommentTok{# home run rate, home runs / at bats}
\end{Highlighting}
\end{Shaded}

\section{Trying to understand the closure
functions}\label{trying-to-understand-the-closure-functions}

Using \texttt{example("function")} is \emph{very} helpful.

\begin{Shaded}
\begin{Highlighting}[]
\NormalTok{is_col <-}\StringTok{ }\KeywordTok{names}\NormalTok{(}\KeywordTok{select_if}\NormalTok{(batpos, is.double))}
\NormalTok{batpos[is_col] <-}\StringTok{ }\NormalTok{batpos[is_col] }\OperatorTok
\StringTok{  }\KeywordTok{map}\NormalTok{(round, }\DataTypeTok{digits =} \DecValTok{4}\NormalTok{)}
\end{Highlighting}
\end{Shaded}

\begin{Shaded}
\begin{Highlighting}[]
\NormalTok{batpos }\OperatorTok\StringTok{ }
\StringTok{  }\KeywordTok{select}\NormalTok{(}\KeywordTok{contains}\NormalTok{(}\StringTok{"B"}\NormalTok{))}
\end{Highlighting}
\end{Shaded}

\begin{verbatim}
## # A tibble: 2,665 x 7
##       AB   X2B   X3B   RBI    SB    BB    BA
##    <int> <int> <int> <int> <int> <int> <dbl>
##  1 12364   624    98  2297   240  1402 0.305
##  2  2044   109    23   242    22   133 0.256
##  3   181     3     0     9     0     6 0.138
##  4  1543    62    19   134    11   288 0.274
##  5  8480   574    59  1363   400  1476 0.291
##  6  2913   180    13   488     8   209 0.294
##  7  2125    94    18   216    31   194 0.241
##  8  4019   163    19   366    29   208 0.254
##  9  2604   107    31   188    25   277 0.269
## 10  1617    79     5   225     6   111 0.280
## # ... with 2,655 more rows
\end{verbatim}

\begin{Shaded}
\begin{Highlighting}[]
\NormalTok{batpos }\OperatorTok\StringTok{ }
\StringTok{  }\KeywordTok{select_all}\NormalTok{(toupper)}
\end{Highlighting}
\end{Shaded}

\begin{verbatim}
## # A tibble: 2,665 x 22
##    PLAYERID   G.X    AB     R     H   X2B   X3B    HR   RBI    SB    CS
##    <chr>    <int> <int> <int> <int> <int> <int> <int> <int> <int> <int>
##  1 aaronha~  3298 12364  2174  3771   624    98   755  2297   240    73
##  2 abbotku~   702  2044   273   523   109    23    62   242    22    11
##  3 abernte~   681   181    12    25     3     0     0     9     0     0
##  4 abramca~   521  1543   246   422    62    19    32   134    11    18
##  5 abreubo~  2425  8480  1453  2470   574    59   288  1363   400   128
##  6 abreujo~   742  2913   398   858   180    13   146   488     8     3
##  7 ackledu~   635  2125   261   512    94    18    46   216    31    12
##  8 adairje~  1165  4019   378  1022   163    19    57   366    29    29
##  9 adamsbo~   797  2604   395   701   107    31    25   188    25    30
## 10 adamsgl~   661  1617   152   452    79     5    34   225     6    10
## # ... with 2,655 more rows, and 11 more variables: BB <int>, SO <int>,
## #   POS <chr>, G.Y <int>, PO <int>, A <int>, E <int>, DP <int>, BA <dbl>,
## #   ERR <dbl>, HRR <dbl>
\end{verbatim}

\begin{Shaded}
\begin{Highlighting}[]
\NormalTok{batpos }\OperatorTok\StringTok{ }
\StringTok{  }\KeywordTok{drop_na}\NormalTok{() }\OperatorTok\StringTok{ }
\StringTok{  }\CommentTok{#select_if(function(x) sum(x == 0) > 100, tolower)}
\StringTok{  }\KeywordTok{select_if}\NormalTok{(}\ControlFlowTok{function}\NormalTok{(x) }\KeywordTok{sum}\NormalTok{(x }\OperatorTok{==}\StringTok{ }\DecValTok{0}\NormalTok{) }\OperatorTok{>}\StringTok{ }\DecValTok{100}\NormalTok{, tolower)}
\end{Highlighting}
\end{Shaded}

\begin{verbatim}
## # A tibble: 2,648 x 8
##      x2b   x3b    hr   rbi    sb    cs    bb     hrr
##    <int> <int> <int> <int> <int> <int> <int>   <dbl>
##  1   624    98   755  2297   240    73  1402 0.0611 
##  2   109    23    62   242    22    11   133 0.0303 
##  3     3     0     0     9     0     0     6 0      
##  4    62    19    32   134    11    18   288 0.0207 
##  5   574    59   288  1363   400   128  1476 0.034  
##  6   180    13   146   488     8     3   209 0.0501 
##  7    94    18    46   216    31    12   194 0.0216 
##  8   163    19    57   366    29    29   208 0.0142 
##  9   107    31    25   188    25    30   277 0.00960
## 10    79     5    34   225     6    10   111 0.021  
## # ... with 2,638 more rows
\end{verbatim}

\begin{Shaded}
\begin{Highlighting}[]
\NormalTok{batpos }\OperatorTok
\StringTok{  }\KeywordTok{drop_na}\NormalTok{() }\OperatorTok\StringTok{ }
\StringTok{  }\KeywordTok{sapply}\NormalTok{(}\ControlFlowTok{function}\NormalTok{(x) }\KeywordTok{sum}\NormalTok{(x }\OperatorTok{==}\StringTok{ }\DecValTok{0}\NormalTok{) }\OperatorTok{>}\StringTok{ }\DecValTok{100}\NormalTok{)}
\end{Highlighting}
\end{Shaded}

\begin{verbatim}
## playerID      G.x       AB        R        H      X2B      X3B       HR 
##    FALSE    FALSE    FALSE    FALSE    FALSE     TRUE     TRUE     TRUE 
##      RBI       SB       CS       BB       SO      POS      G.y       PO 
##     TRUE     TRUE     TRUE     TRUE    FALSE    FALSE    FALSE    FALSE 
##        A        E       DP       BA      Err      HRR 
##    FALSE    FALSE    FALSE    FALSE    FALSE     TRUE
\end{verbatim}

\begin{Shaded}
\begin{Highlighting}[]
\NormalTok{batpos }\OperatorTok\StringTok{ }
\StringTok{  }\KeywordTok{drop_na}\NormalTok{() }\OperatorTok\StringTok{ }
\StringTok{  }\KeywordTok{rename_if}\NormalTok{(}\ControlFlowTok{function}\NormalTok{(x) }\OperatorTok{!}\StringTok{ }\KeywordTok{sum}\NormalTok{(x }\OperatorTok{==}\StringTok{ }\DecValTok{0}\NormalTok{) }\OperatorTok{>}\StringTok{ }\DecValTok{100}\NormalTok{, tolower)}
\end{Highlighting}
\end{Shaded}

\begin{verbatim}
## # A tibble: 2,648 x 22
##    playerid   g.x    ab     r     h   X2B   X3B    HR   RBI    SB    CS
##    <chr>    <int> <int> <int> <int> <int> <int> <int> <int> <int> <int>
##  1 aaronha~  3298 12364  2174  3771   624    98   755  2297   240    73
##  2 abbotku~   702  2044   273   523   109    23    62   242    22    11
##  3 abernte~   681   181    12    25     3     0     0     9     0     0
##  4 abramca~   521  1543   246   422    62    19    32   134    11    18
##  5 abreubo~  2425  8480  1453  2470   574    59   288  1363   400   128
##  6 abreujo~   742  2913   398   858   180    13   146   488     8     3
##  7 ackledu~   635  2125   261   512    94    18    46   216    31    12
##  8 adairje~  1165  4019   378  1022   163    19    57   366    29    29
##  9 adamsbo~   797  2604   395   701   107    31    25   188    25    30
## 10 adamsgl~   661  1617   152   452    79     5    34   225     6    10
## # ... with 2,638 more rows, and 11 more variables: BB <int>, so <int>,
## #   pos <chr>, g.y <int>, po <int>, a <int>, e <int>, dp <int>, ba <dbl>,
## #   err <dbl>, HRR <dbl>
\end{verbatim}

\begin{Shaded}
\begin{Highlighting}[]
\NormalTok{batpos }\OperatorTok\StringTok{ }
\StringTok{  }\KeywordTok{select_at}\NormalTok{(}\KeywordTok{c}\NormalTok{(}\DecValTok{2}\NormalTok{, }\DecValTok{4}\NormalTok{, }\DecValTok{6}\NormalTok{, }\DecValTok{8}\NormalTok{, }\DecValTok{10}\NormalTok{, }\DecValTok{12}\NormalTok{, }\DecValTok{14}\NormalTok{), tolower) }\OperatorTok\StringTok{ }
\StringTok{  }\KeywordTok{rename_at}\NormalTok{(}\KeywordTok{c}\NormalTok{(}\DecValTok{3}\NormalTok{,}\DecValTok{5}\NormalTok{,}\DecValTok{7}\NormalTok{), toupper)}
\end{Highlighting}
\end{Shaded}

\begin{verbatim}
## # A tibble: 2,665 x 7
##      g.x     r   X2B    hr    SB    bb POS  
##    <int> <int> <int> <int> <int> <int> <chr>
##  1  3298  2174   624   755   240  1402 OF   
##  2   702   273   109    62    22   133 SS   
##  3   681    12     3     0     0     6 P    
##  4   521   246    62    32    11   288 OF   
##  5  2425  1453   574   288   400  1476 OF   
##  6   742   398   180   146     8   209 1B   
##  7   635   261    94    46    31   194 2B   
##  8  1165   378   163    57    29   208 2B   
##  9   797   395   107    25    25   277 3B   
## 10   661   152    79    34     6   111 OF   
## # ... with 2,655 more rows
\end{verbatim}

\begin{Shaded}
\begin{Highlighting}[]
\NormalTok{batpos }\OperatorTok\StringTok{ }
\CommentTok{#  select_all(toupper)}
\StringTok{  }\KeywordTok{select_all}\NormalTok{(}\KeywordTok{list}\NormalTok{(}\OperatorTok{~}\StringTok{ }\KeywordTok{toupper}\NormalTok{(.)))}
\end{Highlighting}
\end{Shaded}

\begin{verbatim}
## # A tibble: 2,665 x 22
##    PLAYERID   G.X    AB     R     H   X2B   X3B    HR   RBI    SB    CS
##    <chr>    <int> <int> <int> <int> <int> <int> <int> <int> <int> <int>
##  1 aaronha~  3298 12364  2174  3771   624    98   755  2297   240    73
##  2 abbotku~   702  2044   273   523   109    23    62   242    22    11
##  3 abernte~   681   181    12    25     3     0     0     9     0     0
##  4 abramca~   521  1543   246   422    62    19    32   134    11    18
##  5 abreubo~  2425  8480  1453  2470   574    59   288  1363   400   128
##  6 abreujo~   742  2913   398   858   180    13   146   488     8     3
##  7 ackledu~   635  2125   261   512    94    18    46   216    31    12
##  8 adairje~  1165  4019   378  1022   163    19    57   366    29    29
##  9 adamsbo~   797  2604   395   701   107    31    25   188    25    30
## 10 adamsgl~   661  1617   152   452    79     5    34   225     6    10
## # ... with 2,655 more rows, and 11 more variables: BB <int>, SO <int>,
## #   POS <chr>, G.Y <int>, PO <int>, A <int>, E <int>, DP <int>, BA <dbl>,
## #   ERR <dbl>, HRR <dbl>
\end{verbatim}

\begin{Shaded}
\begin{Highlighting}[]
\NormalTok{batpos }\OperatorTok\StringTok{ }
\CommentTok{#  select_all(toupper)}
\CommentTok{#  select_all(list(~ paste(., "0", sep="")))}
\StringTok{  }\KeywordTok{select_all}\NormalTok{(}\OperatorTok{~}\StringTok{ }\KeywordTok{paste}\NormalTok{(., }\StringTok{"0"}\NormalTok{, }\DataTypeTok{sep=}\StringTok{""}\NormalTok{))}
\end{Highlighting}
\end{Shaded}

\begin{verbatim}
## # A tibble: 2,665 x 22
##    playerID0  G.x0   AB0    R0    H0  X2B0  X3B0   HR0  RBI0   SB0   CS0
##    <chr>     <int> <int> <int> <int> <int> <int> <int> <int> <int> <int>
##  1 aaronha01  3298 12364  2174  3771   624    98   755  2297   240    73
##  2 abbotku01   702  2044   273   523   109    23    62   242    22    11
##  3 abernte02   681   181    12    25     3     0     0     9     0     0
##  4 abramca01   521  1543   246   422    62    19    32   134    11    18
##  5 abreubo01  2425  8480  1453  2470   574    59   288  1363   400   128
##  6 abreujo02   742  2913   398   858   180    13   146   488     8     3
##  7 ackledu01   635  2125   261   512    94    18    46   216    31    12
##  8 adairje01  1165  4019   378  1022   163    19    57   366    29    29
##  9 adamsbo03   797  2604   395   701   107    31    25   188    25    30
## 10 adamsgl01   661  1617   152   452    79     5    34   225     6    10
## # ... with 2,655 more rows, and 11 more variables: BB0 <int>, SO0 <int>,
## #   POS0 <chr>, G.y0 <int>, PO0 <int>, A0 <int>, E0 <int>, DP0 <int>,
## #   BA0 <dbl>, Err0 <dbl>, HRR0 <dbl>
\end{verbatim}

\begin{Shaded}
\begin{Highlighting}[]
\NormalTok{batpos }\OperatorTok\StringTok{ }
\CommentTok{#  select_if(is.numeric, ~ paste(., "new", sep="_"))}
\StringTok{  }\KeywordTok{mutate_if}\NormalTok{(is.numeric, }\ControlFlowTok{function}\NormalTok{(x) }\KeywordTok{log}\NormalTok{(x }\OperatorTok{+}\StringTok{ }\DecValTok{1}\NormalTok{))}
\end{Highlighting}
\end{Shaded}

\begin{verbatim}
## # A tibble: 2,665 x 22
##    playerID   G.x    AB     R     H   X2B   X3B    HR   RBI    SB    CS
##    <chr>    <dbl> <dbl> <dbl> <dbl> <dbl> <dbl> <dbl> <dbl> <dbl> <dbl>
##  1 aaronha~  8.10  9.42  7.68  8.24  6.44  4.60  6.63  7.74  5.48  4.30
##  2 abbotku~  6.56  7.62  5.61  6.26  4.70  3.18  4.14  5.49  3.14  2.48
##  3 abernte~  6.53  5.20  2.56  3.26  1.39  0     0     2.30  0     0   
##  4 abramca~  6.26  7.34  5.51  6.05  4.14  3.00  3.50  4.91  2.48  2.94
##  5 abreubo~  7.79  9.05  7.28  7.81  6.35  4.09  5.67  7.22  5.99  4.86
##  6 abreujo~  6.61  7.98  5.99  6.76  5.20  2.64  4.99  6.19  2.20  1.39
##  7 ackledu~  6.46  7.66  5.57  6.24  4.55  2.94  3.85  5.38  3.47  2.56
##  8 adairje~  7.06  8.30  5.94  6.93  5.10  3.00  4.06  5.91  3.40  3.40
##  9 adamsbo~  6.68  7.87  5.98  6.55  4.68  3.47  3.26  5.24  3.26  3.43
## 10 adamsgl~  6.50  7.39  5.03  6.12  4.38  1.79  3.56  5.42  1.95  2.40
## # ... with 2,655 more rows, and 11 more variables: BB <dbl>, SO <dbl>,
## #   POS <chr>, G.y <dbl>, PO <dbl>, A <dbl>, E <dbl>, DP <dbl>, BA <dbl>,
## #   Err <dbl>, HRR <dbl>
\end{verbatim}

\begin{Shaded}
\begin{Highlighting}[]
\NormalTok{batpos }\OperatorTok\StringTok{ }
\StringTok{  }\KeywordTok{rename_if}\NormalTok{(is.numeric, }\OperatorTok{~}\StringTok{ }\KeywordTok{paste}\NormalTok{(., }\StringTok{"N"}\NormalTok{, }\DataTypeTok{sep=}\StringTok{""}\NormalTok{))}
\end{Highlighting}
\end{Shaded}

\begin{verbatim}
## # A tibble: 2,665 x 22
##    playerID  G.xN   ABN    RN    HN  X2BN  X3BN   HRN  RBIN   SBN   CSN
##    <chr>    <int> <int> <int> <int> <int> <int> <int> <int> <int> <int>
##  1 aaronha~  3298 12364  2174  3771   624    98   755  2297   240    73
##  2 abbotku~   702  2044   273   523   109    23    62   242    22    11
##  3 abernte~   681   181    12    25     3     0     0     9     0     0
##  4 abramca~   521  1543   246   422    62    19    32   134    11    18
##  5 abreubo~  2425  8480  1453  2470   574    59   288  1363   400   128
##  6 abreujo~   742  2913   398   858   180    13   146   488     8     3
##  7 ackledu~   635  2125   261   512    94    18    46   216    31    12
##  8 adairje~  1165  4019   378  1022   163    19    57   366    29    29
##  9 adamsbo~   797  2604   395   701   107    31    25   188    25    30
## 10 adamsgl~   661  1617   152   452    79     5    34   225     6    10
## # ... with 2,655 more rows, and 11 more variables: BBN <int>, SON <int>,
## #   POS <chr>, G.yN <int>, PON <int>, AN <int>, EN <int>, DPN <int>,
## #   BAN <dbl>, ErrN <dbl>, HRRN <dbl>
\end{verbatim}

\begin{Shaded}
\begin{Highlighting}[]
\NormalTok{batpos }\OperatorTok\StringTok{ }
\StringTok{  }\KeywordTok{rename_at}\NormalTok{(}\KeywordTok{vars}\NormalTok{(}\KeywordTok{contains}\NormalTok{(}\StringTok{"B"}\NormalTok{)), }\OperatorTok{~}\StringTok{ }\KeywordTok{tolower}\NormalTok{(.))}
\end{Highlighting}
\end{Shaded}

\begin{verbatim}
## # A tibble: 2,665 x 22
##    playerID   G.x    ab     R     H   x2b   x3b    HR   rbi    sb    CS
##    <chr>    <int> <int> <int> <int> <int> <int> <int> <int> <int> <int>
##  1 aaronha~  3298 12364  2174  3771   624    98   755  2297   240    73
##  2 abbotku~   702  2044   273   523   109    23    62   242    22    11
##  3 abernte~   681   181    12    25     3     0     0     9     0     0
##  4 abramca~   521  1543   246   422    62    19    32   134    11    18
##  5 abreubo~  2425  8480  1453  2470   574    59   288  1363   400   128
##  6 abreujo~   742  2913   398   858   180    13   146   488     8     3
##  7 ackledu~   635  2125   261   512    94    18    46   216    31    12
##  8 adairje~  1165  4019   378  1022   163    19    57   366    29    29
##  9 adamsbo~   797  2604   395   701   107    31    25   188    25    30
## 10 adamsgl~   661  1617   152   452    79     5    34   225     6    10
## # ... with 2,655 more rows, and 11 more variables: bb <int>, SO <int>,
## #   POS <chr>, G.y <int>, PO <int>, A <int>, E <int>, DP <int>, ba <dbl>,
## #   Err <dbl>, HRR <dbl>
\end{verbatim}

\begin{Shaded}
\begin{Highlighting}[]
\NormalTok{batpos }\OperatorTok\StringTok{ }
\StringTok{  }\KeywordTok{select}\NormalTok{(}\KeywordTok{contains}\NormalTok{(}\StringTok{"B"}\NormalTok{)) }\OperatorTok\StringTok{ }
\StringTok{  }\KeywordTok{rename_all}\NormalTok{(}\OperatorTok{~}\StringTok{ }\KeywordTok{tolower}\NormalTok{(.))}
\end{Highlighting}
\end{Shaded}

\begin{verbatim}
## # A tibble: 2,665 x 7
##       ab   x2b   x3b   rbi    sb    bb    ba
##    <int> <int> <int> <int> <int> <int> <dbl>
##  1 12364   624    98  2297   240  1402 0.305
##  2  2044   109    23   242    22   133 0.256
##  3   181     3     0     9     0     6 0.138
##  4  1543    62    19   134    11   288 0.274
##  5  8480   574    59  1363   400  1476 0.291
##  6  2913   180    13   488     8   209 0.294
##  7  2125    94    18   216    31   194 0.241
##  8  4019   163    19   366    29   208 0.254
##  9  2604   107    31   188    25   277 0.269
## 10  1617    79     5   225     6   111 0.280
## # ... with 2,655 more rows
\end{verbatim}

\begin{Shaded}
\begin{Highlighting}[]
\CommentTok{# or}
\NormalTok{batpos }\OperatorTok\StringTok{ }
\StringTok{  }\KeywordTok{select_at}\NormalTok{(}\KeywordTok{vars}\NormalTok{(}\KeywordTok{contains}\NormalTok{(}\StringTok{"B"}\NormalTok{)), }\OperatorTok{~}\StringTok{ }\KeywordTok{tolower}\NormalTok{(.))}
\end{Highlighting}
\end{Shaded}

\begin{verbatim}
## # A tibble: 2,665 x 7
##       ab   x2b   x3b   rbi    sb    bb    ba
##    <int> <int> <int> <int> <int> <int> <dbl>
##  1 12364   624    98  2297   240  1402 0.305
##  2  2044   109    23   242    22   133 0.256
##  3   181     3     0     9     0     6 0.138
##  4  1543    62    19   134    11   288 0.274
##  5  8480   574    59  1363   400  1476 0.291
##  6  2913   180    13   488     8   209 0.294
##  7  2125    94    18   216    31   194 0.241
##  8  4019   163    19   366    29   208 0.254
##  9  2604   107    31   188    25   277 0.269
## 10  1617    79     5   225     6   111 0.280
## # ... with 2,655 more rows
\end{verbatim}

\begin{Shaded}
\begin{Highlighting}[]
\NormalTok{batpos }\OperatorTok\StringTok{ }
\StringTok{  }\KeywordTok{keep}\NormalTok{(is.numeric) }\OperatorTok\StringTok{ }
\StringTok{  }\KeywordTok{filter_all}\NormalTok{(}\KeywordTok{all_vars}\NormalTok{(. }\OperatorTok{<}\StringTok{ }\DecValTok{1000}\NormalTok{))}
\end{Highlighting}
\end{Shaded}

\begin{verbatim}
## # A tibble: 316 x 20
##      G.x    AB     R     H   X2B   X3B    HR   RBI    SB    CS    BB    SO
##    <int> <int> <int> <int> <int> <int> <int> <int> <int> <int> <int> <int>
##  1   681   181    12    25     3     0     0     9     0     0     6    74
##  2   574    78     2     4     1     0     0     2     0     0     7    41
##  3   774    17     0     3     0     0     0     2     0     0     2     6
##  4   543    20     0     2     0     0     0     0     0     0     1     7
##  5   737   139    12    28     3     0     3    11     0     0     6    37
##  6   549    35     1     3     1     0     0     0     0     0     0    21
##  7   562   265    19    44     8     0     0    17     0     0     9    77
##  8   592    14     0     2     0     0     0     2     0     0     0     8
##  9   699    38     4     5     0     0     0     0     0     0     3    15
## 10   884    36     3     3     1     0     0     0     0     0     5    12
## # ... with 306 more rows, and 8 more variables: G.y <int>, PO <int>,
## #   A <int>, E <int>, DP <int>, BA <dbl>, Err <dbl>, HRR <dbl>
\end{verbatim}

\begin{Shaded}
\begin{Highlighting}[]
\NormalTok{batpos }\OperatorTok\StringTok{ }
\StringTok{  }\KeywordTok{filter_all}\NormalTok{(}\KeywordTok{any_vars}\NormalTok{(. }\OperatorTok{>}\StringTok{ }\DecValTok{10000}\NormalTok{))}
\end{Highlighting}
\end{Shaded}

\begin{verbatim}
## # A tibble: 2,665 x 22
##    playerID   G.x    AB     R     H   X2B   X3B    HR   RBI    SB    CS
##    <chr>    <int> <int> <int> <int> <int> <int> <int> <int> <int> <int>
##  1 aaronha~  3298 12364  2174  3771   624    98   755  2297   240    73
##  2 abbotku~   702  2044   273   523   109    23    62   242    22    11
##  3 abernte~   681   181    12    25     3     0     0     9     0     0
##  4 abramca~   521  1543   246   422    62    19    32   134    11    18
##  5 abreubo~  2425  8480  1453  2470   574    59   288  1363   400   128
##  6 abreujo~   742  2913   398   858   180    13   146   488     8     3
##  7 ackledu~   635  2125   261   512    94    18    46   216    31    12
##  8 adairje~  1165  4019   378  1022   163    19    57   366    29    29
##  9 adamsbo~   797  2604   395   701   107    31    25   188    25    30
## 10 adamsgl~   661  1617   152   452    79     5    34   225     6    10
## # ... with 2,655 more rows, and 11 more variables: BB <int>, SO <int>,
## #   POS <chr>, G.y <int>, PO <int>, A <int>, E <int>, DP <int>, BA <dbl>,
## #   Err <dbl>, HRR <dbl>
\end{verbatim}

\begin{Shaded}
\begin{Highlighting}[]
\NormalTok{batpos }\OperatorTok\StringTok{ }
\StringTok{  }\KeywordTok{filter_if}\NormalTok{(is.numeric, }\KeywordTok{all_vars}\NormalTok{(. }\OperatorTok{<}\StringTok{ }\DecValTok{600}\NormalTok{))}
\end{Highlighting}
\end{Shaded}

\begin{verbatim}
## # A tibble: 121 x 22
##    playerID   G.x    AB     R     H   X2B   X3B    HR   RBI    SB    CS
##    <chr>    <int> <int> <int> <int> <int> <int> <int> <int> <int> <int>
##  1 adamste~   574    78     2     4     1     0     0     2     0     0
##  2 agostju~   543    20     0     2     0     0     0     0     0     0
##  3 alberma~   549    35     1     3     1     0     0     0     0     0
##  4 alexado~   562   265    19    44     8     0     0    17     0     0
##  5 alfonan~   592    14     0     2     0     0     0     2     0     0
##  6 axforjo~   543     1     0     0     0     0     0     0     0     0
##  7 ayalalu~   534    14     0     4     1     0     0     0     0     0
##  8 baezda01   533     6     1     1     0     0     0     0     0     0
##  9 bahnsst~   575   479    22    56     8     2     1    19     0     0
## 10 bairdo01   584    52     2     5     1     0     1     4     0     0
## # ... with 111 more rows, and 11 more variables: BB <int>, SO <int>,
## #   POS <chr>, G.y <int>, PO <int>, A <int>, E <int>, DP <int>, BA <dbl>,
## #   Err <dbl>, HRR <dbl>
\end{verbatim}

\begin{Shaded}
\begin{Highlighting}[]
\NormalTok{batpos }\OperatorTok\StringTok{ }
\StringTok{  }\KeywordTok{select_at}\NormalTok{(}\DecValTok{4}\OperatorTok{:}\DecValTok{10}\NormalTok{)}
\end{Highlighting}
\end{Shaded}

\begin{verbatim}
## # A tibble: 2,665 x 7
##        R     H   X2B   X3B    HR   RBI    SB
##    <int> <int> <int> <int> <int> <int> <int>
##  1  2174  3771   624    98   755  2297   240
##  2   273   523   109    23    62   242    22
##  3    12    25     3     0     0     9     0
##  4   246   422    62    19    32   134    11
##  5  1453  2470   574    59   288  1363   400
##  6   398   858   180    13   146   488     8
##  7   261   512    94    18    46   216    31
##  8   378  1022   163    19    57   366    29
##  9   395   701   107    31    25   188    25
## 10   152   452    79     5    34   225     6
## # ... with 2,655 more rows
\end{verbatim}

\begin{Shaded}
\begin{Highlighting}[]
\NormalTok{batpos }\OperatorTok\StringTok{ }
\StringTok{  }\KeywordTok{filter_at}\NormalTok{(}\DecValTok{4}\OperatorTok{:}\DecValTok{6}\NormalTok{, }\KeywordTok{all_vars}\NormalTok{((. }\OperatorTok\StringTok{ }\DecValTok{10}\NormalTok{) }\OperatorTok{==}\StringTok{ }\DecValTok{5}\NormalTok{))}
\end{Highlighting}
\end{Shaded}

\begin{verbatim}
## # A tibble: 3 x 22
##   playerID   G.x    AB     R     H   X2B   X3B    HR   RBI    SB    CS
##   <chr>    <int> <int> <int> <int> <int> <int> <int> <int> <int> <int>
## 1 freemfr~  1188  4356   685  1275   285    20   189   684    37    18
## 2 wrighta~  1029  3583   465  1115   175    55    38   553    32    33
## 3 wynnji01  1920  6653  1105  1665   285    39   291   964   225   101
## # ... with 11 more variables: BB <int>, SO <int>, POS <chr>, G.y <int>,
## #   PO <int>, A <int>, E <int>, DP <int>, BA <dbl>, Err <dbl>, HRR <dbl>
\end{verbatim}

\begin{Shaded}
\begin{Highlighting}[]
\NormalTok{batpos }\OperatorTok\StringTok{ }
\StringTok{  }\KeywordTok{filter_at}\NormalTok{(}\KeywordTok{vars}\NormalTok{(}\KeywordTok{starts_with}\NormalTok{(}\StringTok{"X"}\NormalTok{)), }\KeywordTok{any_vars}\NormalTok{((. }\OperatorTok\StringTok{ }\DecValTok{50}\NormalTok{) }\OperatorTok{==}\StringTok{ }\DecValTok{0} \OperatorTok{&}\StringTok{ }\NormalTok{. }\OperatorTok{>}\StringTok{ }\DecValTok{0}\NormalTok{))}
\end{Highlighting}
\end{Shaded}

\begin{verbatim}
## # A tibble: 74 x 22
##    playerID   G.x    AB     R     H   X2B   X3B    HR   RBI    SB    CS
##    <chr>    <int> <int> <int> <int> <int> <int> <int> <int> <int> <int>
##  1 alexama~   594  1271   166   293    50    12    15   115    37    10
##  2 alouma01  1667  5789   780  1777   236    50    31   427   156    80
##  3 batteea~  1141  3586   393   969   150    17   104   449    13    12
##  4 beckeri~   789  2227   345   570   100    12    45   243    66    26
##  5 bergmda~  1349  2679   312   690   100    16    54   289    19    14
##  6 berryke~  1383  4136   422  1053   150    23    58   343    45    46
##  7 bigbeca~   712  2703   443   826   100    55    12   250   103    68
##  8 bochtbr~  1538  5233   643  1478   250    21   100   658    43    41
##  9 bosleth~   784  1581   183   430    50    12    20   158    47    24
## 10 boyercl~  1725  5780   645  1396   200    33   162   654    41    28
## # ... with 64 more rows, and 11 more variables: BB <int>, SO <int>,
## #   POS <chr>, G.y <int>, PO <int>, A <int>, E <int>, DP <int>, BA <dbl>,
## #   Err <dbl>, HRR <dbl>
\end{verbatim}

\begin{Shaded}
\begin{Highlighting}[]
\NormalTok{is_whole <-}\StringTok{ }\ControlFlowTok{function}\NormalTok{(x) }\ControlFlowTok{if}\NormalTok{(}\KeywordTok{is.numeric}\NormalTok{(x)) }\KeywordTok{all}\NormalTok{(}\KeywordTok{floor}\NormalTok{(x) }\OperatorTok{==}\StringTok{ }\NormalTok{x) }\ControlFlowTok{else} \OtherTok{FALSE}
\CommentTok{#batpos %>% }
\CommentTok{#  keep(is_whole) %>% }
\CommentTok{#  filter_if(~ all(floor(.) == .), any_vars((. %% 100) == 50))}
\NormalTok{batpos }\OperatorTok\StringTok{ }
\StringTok{  }\KeywordTok{filter_if}\NormalTok{(is_whole, }\KeywordTok{any_vars}\NormalTok{((. }\OperatorTok\StringTok{ }\DecValTok{100}\NormalTok{) }\OperatorTok{==}\StringTok{ }\DecValTok{50}\NormalTok{))}
\end{Highlighting}
\end{Shaded}

\begin{verbatim}
## # A tibble: 380 x 22
##    playerID   G.x    AB     R     H   X2B   X3B    HR   RBI    SB    CS
##    <chr>    <int> <int> <int> <int> <int> <int> <int> <int> <int> <int>
##  1 adamsma~   707  2026   248   539   113     6    96   332     4     4
##  2 alexada~   662  2450   369   811   164    30    61   459    20    28
##  3 alexama~   594  1271   166   293    50    12    15   115    37    10
##  4 alicelu~  1341  3971   551  1031   189    53    47   422    81    50
##  5 allisbo~  1541  5032   811  1281   216    53   256   796    84    50
##  6 almonbi~  1236  3330   390   846   138    25    36   296   128    60
##  7 alouje01  1380  4345   448  1216   170    26    32   377    31    46
##  8 alouma01  1667  5789   780  1777   236    50    31   427   156    80
##  9 amarial~   702  1750   171   404    67    16    21   169    39    10
## 10 aurilri~  1652  5721   745  1576   301    22   186   756    23    18
## # ... with 370 more rows, and 11 more variables: BB <int>, SO <int>,
## #   POS <chr>, G.y <int>, PO <int>, A <int>, E <int>, DP <int>, BA <dbl>,
## #   Err <dbl>, HRR <dbl>
\end{verbatim}

\section{Tidy evaluation with rlang}\label{tidy-evaluation-with-rlang}

Symbols:

\begin{Shaded}
\begin{Highlighting}[]
\KeywordTok{library}\NormalTok{(rlang)}
\KeywordTok{cat}\NormalTok{(pi, }\KeywordTok{expr}\NormalTok{(pi), }\KeywordTok{eval}\NormalTok{(}\KeywordTok{expr}\NormalTok{(pi)), }\StringTok{'}\CharTok{\textbackslash{}n}\StringTok{'}\NormalTok{)}
\end{Highlighting}
\end{Shaded}

\begin{verbatim}
## 3.141593 pi 3.141593
\end{verbatim}

\begin{Shaded}
\begin{Highlighting}[]
\KeywordTok{cat}\NormalTok{(}\KeywordTok{is_symbol}\NormalTok{(pi), }\KeywordTok{is_symbol}\NormalTok{(}\KeywordTok{expr}\NormalTok{(pi)))}
\end{Highlighting}
\end{Shaded}

\begin{verbatim}
## FALSE TRUE
\end{verbatim}

\begin{Shaded}
\begin{Highlighting}[]
\NormalTok{print_types <-}\StringTok{ }\ControlFlowTok{function}\NormalTok{(x) \{}
  \KeywordTok{print}\NormalTok{(x)}
  \KeywordTok{print}\NormalTok{(}\KeywordTok{eval}\NormalTok{(x))}
  \KeywordTok{cat}\NormalTok{(}\StringTok{'  Symbol:'}\NormalTok{, }\KeywordTok{is_symbol}\NormalTok{(x))}
  \KeywordTok{cat}\NormalTok{(}\StringTok{' Environment:'}\NormalTok{, }\KeywordTok{is_environment}\NormalTok{(x))}
  \KeywordTok{cat}\NormalTok{(}\StringTok{' Constant:'}\NormalTok{, }\KeywordTok{is_bare_atomic}\NormalTok{(x))}
  \KeywordTok{cat}\NormalTok{(}\StringTok{'}\CharTok{\textbackslash{}n}\StringTok{  Call object:'}\NormalTok{, }\KeywordTok{is_call}\NormalTok{(x))}
  \KeywordTok{cat}\NormalTok{(}\StringTok{' Expression:'}\NormalTok{, }\KeywordTok{is_expression}\NormalTok{(x))}
  \KeywordTok{cat}\NormalTok{(}\StringTok{' Quosure:'}\NormalTok{, }\KeywordTok{is_quosure}\NormalTok{(x))}
  \KeywordTok{cat}\NormalTok{(}\StringTok{'}\CharTok{\textbackslash{}n}\StringTok{'}\NormalTok{)}
\NormalTok{\}}
\end{Highlighting}
\end{Shaded}

\begin{Shaded}
\begin{Highlighting}[]
\NormalTok{a <-}\StringTok{ }\DecValTok{1}
\NormalTok{b <-}\StringTok{ }\DecValTok{2}
\KeywordTok{sapply}\NormalTok{(}\KeywordTok{c}\NormalTok{(pi, }\DecValTok{1}\NormalTok{, }\KeywordTok{abs}\NormalTok{(}\DecValTok{1}\NormalTok{), pi, }\KeywordTok{expr}\NormalTok{(pi), }\KeywordTok{expr}\NormalTok{(a}\OperatorTok{+}\NormalTok{b), }\KeywordTok{quo}\NormalTok{(a}\OperatorTok{+}\NormalTok{b)),}
\NormalTok{       print_types)}
\end{Highlighting}
\end{Shaded}

\begin{verbatim}
## [1] 3.141593
## [1] 3.141593
##   Symbol: FALSE Environment: FALSE Constant: TRUE
##   Call object: FALSE Expression: TRUE Quosure: FALSE
## [1] 1
## [1] 1
##   Symbol: FALSE Environment: FALSE Constant: TRUE
##   Call object: FALSE Expression: TRUE Quosure: FALSE
## [1] 1
## [1] 1
##   Symbol: FALSE Environment: FALSE Constant: TRUE
##   Call object: FALSE Expression: TRUE Quosure: FALSE
## [1] 3.141593
## [1] 3.141593
##   Symbol: FALSE Environment: FALSE Constant: TRUE
##   Call object: FALSE Expression: TRUE Quosure: FALSE
## pi
## [1] 3.141593
##   Symbol: TRUE Environment: FALSE Constant: FALSE
##   Call object: FALSE Expression: TRUE Quosure: FALSE
## a + b
## [1] 3
##   Symbol: FALSE Environment: FALSE Constant: FALSE
##   Call object: TRUE Expression: TRUE Quosure: FALSE
## <quosure>
## expr: ^a + b
## env:  global
## <quosure>
## expr: ^a + b
## env:  global
##   Symbol: FALSE Environment: FALSE Constant: FALSE
##   Call object: TRUE Expression: TRUE Quosure: TRUE
\end{verbatim}

\begin{verbatim}
## [[1]]
## NULL
## 
## [[2]]
## NULL
## 
## [[3]]
## NULL
## 
## [[4]]
## NULL
## 
## [[5]]
## NULL
## 
## [[6]]
## NULL
## 
## [[7]]
## NULL
\end{verbatim}

\begin{Shaded}
\begin{Highlighting}[]
\KeywordTok{quos}\NormalTok{(a}\OperatorTok{+}\NormalTok{b, a}\OperatorTok{-}\NormalTok{b)}
\end{Highlighting}
\end{Shaded}

\begin{verbatim}
## <list_of<quosure>>
## 
## [[1]]
## <quosure>
## expr: ^a + b
## env:  global
## 
## [[2]]
## <quosure>
## expr: ^a - b
## env:  global
\end{verbatim}

\begin{Shaded}
\begin{Highlighting}[]
\NormalTok{quote_this <-}\StringTok{ }\ControlFlowTok{function}\NormalTok{(x) }\KeywordTok{enquo}\NormalTok{(x)}
\NormalTok{quote_these <-}\StringTok{ }\ControlFlowTok{function}\NormalTok{(...) }\KeywordTok{enquos}\NormalTok{(...)}
\end{Highlighting}
\end{Shaded}

\begin{Shaded}
\begin{Highlighting}[]
\CommentTok{# quosures allow code to be written from string variables}
\CommentTok{# and vice versa}
\KeywordTok{print}\NormalTok{(}\DecValTok{1} \OperatorTok{+}\StringTok{ }\KeywordTok{eval}\NormalTok{(}\KeywordTok{parse_expr}\NormalTok{(}\StringTok{"a + b"}\NormalTok{)))}
\end{Highlighting}
\end{Shaded}

\begin{verbatim}
## [1] 4
\end{verbatim}

\begin{Shaded}
\begin{Highlighting}[]
\KeywordTok{print}\NormalTok{(}\KeywordTok{expr_text}\NormalTok{(}\ControlFlowTok{function}\NormalTok{(x) x}\OperatorTok{^}\DecValTok{2}\NormalTok{))}
\end{Highlighting}
\end{Shaded}

\begin{verbatim}
## [1] "function (x) \nx^2"
\end{verbatim}

\chapter{caret Functionality}\label{ch:caret}

Using \emph{Applied Predictive Modeling} \citep{kuhn2013applied}.

\chapter{the Machine Learning with R package}\label{ch:mlr}

Using \texttt{mlr} \citep{mlr}.

\emph{more to come}

\chapter{implementing neural networks in R}\label{ch:keras}

Using \texttt{keras}.

\emph{more to come}

\chapter{Tips and tricks}\label{ch:tips}

\emph{more to come}

\bibliography{../clanker.bib,packages.bib}


\end{document}
